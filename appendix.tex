\appendix
\chapter{Installation}
\label{chap:installation}

% TODO: Pull-in eppex to the upstream Moses repository?
% \emph{Eppex} is shipped along with the Moses system and can be obtained the
% same way: by checking out Moses repository at Github.

\section*{Prerequisities}

Only Boost is required to compile and run \eppex{}.
If you had successfully compiled Moses then you probably already have all you need in place.

Just in case you did not want to install Boost as a whole, but only the necessary
parts of it, you will need to get two libraries (build them or install
them via your packaging system): \emph{Program Options} and \emph{IOStreams}.
\Eppex{} also requires working include path access to following
header-only libraries: \emph{Iterator}, \emph{Pool} and \emph{Tokenizer}.

Besides that you will probably like to employ the power of hash tables
that comes with the recent version of C++ standard\footurl{http://en.cppreference.com/w/cpp/container/unordered_set}.
Most of decent C++ compilers will have this feature available.

\section*{Download}

\emph{Eppex} is hosted on Github\footurl{http://www.github.com} in
a fork\footurl{https://github.com/chesio/mosesdecoder}
of core Moses repository\footurl{https://github.com/moses-smt/mosesdecoder}.
If you have your own local clone of Moses repo then all you need is to add
the fork as a new remote, fetch from it and checkout the remote branch
\texttt{eppex} into some new local branch.
\begin{verbatim}
 cd <your-mosesdecoder-dir>
 git remote add eppex https://github.com/chesio/mosesdecoder.git
 git fetch eppex
 git checkout -b eppex eppex:eppex
\end{verbatim}

Important note: \eppex{} branch is based on \emph{RELEASE-1.0},
not the current master. Should you like to use \eppex{} with the most
recent Moses code, you may try merging branch \texttt{master} into \texttt{eppex}:
\begin{verbatim}
 git merge master
\end{verbatim}

Be warned that you may encounter some merge conflicts in \emph{train-model.perl}
as this script gets frequently updated.
In such case feel free to contact the maintainer of \eppex{},
he will be happy to help you.

\section*{Installation}

\Eppex{} related files may be found in \texttt{contrib/eppex} directory of \eppex{} fork
and for the time being it has to be compiled separately from the Moses
build.\footnote{For the build instructions on Moses consult the file \texttt{BUILD-INSTRUCTIONS.txt}.}
Once compiled you will find \eppex{} binary (or symbolic link to it) in the
\texttt{contrib/eppex} directory. You are free to move it to whatever location
suits your working environment, just do not forget to pass the full path
to \emph{train-model.perl} script invocation.

\subsection*{Installation using Boost.Build}

This approach is recommended since \emph{Boost.Build}\footurl{http://www.boost.org/boost-build2/doc/html/index.html}
is required by Moses install process as well.
If you would like to compile \eppex{} with hash tables implementation just pass \texttt{--with-hashtables} flag
to \emph{bjam} command:
\begin{verbatim}
  cd contrib/eppex
  bjam [--with-hashtables]
\end{verbatim}

\subsection*{Installation using make}
The \texttt{contrib/eppex} directory also contains a Makefile that may be used to compile
\emph{eppex} and should work on most Linux systems that have gcc compiler installed:
\begin{verbatim}
  cd contrib/eppex
  make all
\end{verbatim}
