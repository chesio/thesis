% In this chapter:
% - introduction of eppex
% -- design goals
% -- distinction from legacy phrase-extract tools
% - implementation details
% -- Murmur hash, boost pools, std c++11 hash tables, indexed storages, gzipped I/O
% - usage
% -- counting, extracting and scoring mode
% -- missing features

\chapter{Eppex}
\label{chap:eppex}

\Eppex{} is phrase pairs extraction and scoring tool capable of obtaining
approximate frequency counts of extracted phrase pairs via Lossy Counting algorithm
(thus the name \eppex{}, an acronym for \emph{epochal phrase pairs extraction}).
It is designed to be an alternative to standard phrase extraction and scoring tools,
implementing most of the functionality of steps 5 and 6 of \emph{train-model.perl} script.
\Eppex{} input and output interface is fully compatible with those of the replaced tools
and \eppex{} in fact is intended to be invoked from within the training script
by passing specific parameters to \emph{train-model.perl}.

\Eppex{} differs from its core counterparts in one important aspect: during its
runtime only physical memory is utilized, no temporary files are stored on disk
as with \emph{extract} and \emph{score} tools.
The goal is to make \eppex{} a faster alternative, aiming at environments with plenty of RAM.
Benchmarked time/memory trade-offs are fundamental part of this work and
are thoroughly examined in \Cref{chap:results}.

\Eppex{} memory demands may be limited by setting more restrictive support and error thresholds
for Lossy Counting, but aggressive pruning may lead to significant loss of phrase table quality
-- experimentally evaluated trade-offs are also discussed in \Cref{chap:results}.

\section{Implementation}

\Eppex{} is implemented as command-line program and it is written in C++.
Similarly to Moses, its primary target platform is Linux system with \emph{gcc} tool set,
but its implementation does not employ any platform-dependent functionality,
so porting should be rather straight-forward.
For details of how to install \eppex{} please refer to \Aref{chap:installation}.
In short: if your system have Boost library set up and a recent version of C++ compiler
-- one with STL implementation of hash tables introduced with \emph{C++11} standard --
you should have no problems installing and running \eppex{}.

% TODO: Info on GCC versions with std::unordered_ implementation.

% TODO: Mention implementation tricks and tweaks:
% Murmur hash, boost pools, std c++11 hash tables, indexed storages, gzipped I/O
