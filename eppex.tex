\chapter{Eppex}
\label{chap:eppex}

\Eppex{} is phrase pairs extraction and scoring tool capable of obtaining
approximate frequency counts of extracted phrase pairs via Lossy Counting algorithm
(thus the name \eppex{}, an acronym for \emph{epochal phrase pairs extraction}).
It is designed to be an alternative to standard phrase extraction and scoring tools,
replacing most of the functionality of steps 5 and 6 of \emph{train-model.perl} script.
\Eppex{} input and output interface is fully compatible with those of the replaced tools.

During \eppex{} runtime only physical memory is utilized, so in general it
does its job considerably faster than default phrase extraction and scoring components.
On the other hand much more of physical memory is needed on processing machine.
Benchmarked time/memory trade-offs are further examined in \Cref{chap:results}.

\Eppex{} memory demands may be limited by setting more restrictive support and error thresholds,
but aggressive pruning may lead to significant loss of phrase table quality --
experimentally measured trade-offs are also discussed in \Cref{chap:results}.

\section{Implementation}

\emph{Eppex} is implemented as command-line program written in C++.
Similarly to Moses, its primary target platform is Linux system with \emph{gcc} toolset,
but its implementation does not employ any platform-dependent functionality,
so porting should be rather straight-forward.
For details of how to install \eppex{} please refer to \Aref{chap:installation}.
In short: if your system have Boost library set up and a recent version of C++ compiler
-- one with STL implementation of hash tables introduced with \emph{C++11} standard --
you should have no problems installing and running \eppex{}.

% TODO: Info on GCC versions with std::unordered_ implementation.

% TODO: Mention implementation tricks and tweaks:
% Murmur hash, boost pools, std c++11 hash tables, indexed storages, gzipped I/O

\section{Usage}

As mentioned above, \emph{eppex} is a command line tool.

\Eppex{} requires three input files:
the target side part of corpus, the source side part of corpus and the alignment information.
Depending on your goal, you may use \eppex{} to:
\begin{itemize}
 \item collect aggregated counts of phrase pairs of same lengths
 \item generate direct and inverse extract files (as alternative to \emph{extract} tool)
 \item construct phrase table (as alternative to both \emph{extract} and \emph{score} tools)
\end{itemize}

Eppex is a drop-in replacement -- it expects to be fed the same input files
as \emph{extract} component of \emph{phrase-extract} toolset and produces
the same output file as scoring components.

\subsection{Terminology}

% Introduce the definitions of:
% - phrase pair length
% - positive and negative lossy counting limits
% - lambda parameter

A phrase pair length is defined as the length of its longest compound.

A more comfortable way of spe

\subsection{Command line options}

It can be used to produce the following output:
\begin{itemize}
 \item \emph{counts file} - file with frequency counts for each phrase pair
  length up to given limit (set via \verb|--max-phrase-len| option)
 \item \emph{extract files} - direct and inverse extract files. These files are the
  same as the output files produced in step 5 of \emph{train-model.perl} pipeline,
  except that the frequency count of each phrase pairs is not determined by
  the number of times it occurs in the files but for space efficiency it is
  dumped as an additional, 4-th field of each phrase pair record (this format
  is properly recognized by \emph{phrase-extract/scorer}).
 \item \emph{phrase table file} - the ultimate phrase table.
\end{itemize}

The command line syntax for \emph{eppex} is following:

\begin{verbatim}
  ./eppex tgt src align [dir-lex-table [inv-lex-table]] <options>
\end{verbatim}

The positional arguments are:
\begin{itemize}
 \item \verb|tgt| - path to target language corpus (required).
  May be also provided via \verb|--tgt| option.
 \item \verb|src| - path to source language corpus (required).
  May be also provided via \verb|--src| option.
 \item \verb|align| - path to alignments file (required).
  May be also provided via \verb|--align| option.
 \item \verb|dir-lex-table| - path to direct lexical table (optional).
  May be also provided via \verb|--dir-lex-table| option. 
 \item \verb|inv-lex-table| - path to inverse lexical table (optional).
  May be also provided via \verb|--inv-lex-table| option. 
\end{itemize}

Options related to phrase pairs counting:
\begin{itemize}
 \item \verb|--counts-file| - path to file with phrase pairs counts.
 \item \verb|--reuse-counts| - use counts from the provided file instead
  of counting anew (skips counting mode).
 \item \verb|--max-phrase-len| - maximum length of extracted phrases.
\end{itemize}

Options related to phrase pairs extraction:
\begin{itemize}
 \item \verb|--extract-file| - path to file for extracted phrase pairs.
  The inverse file has $.inv$ extension attached to its name automatically.
 \item \verb|--thresholds| - a space separated list of \emph{error} and \emph{support}
  thresholds for lossy counting. For each phrase pair length or range of lengths a separate lossy counter
  can be specified, for example: 1:0.5:1.5 2-3:1.0:2.0 4-7:3.0:5.0
 \item \verb|--limits| - a space separated list of \emph{negative} and \emph{positive}
  limits for lossy counting. For each phrase pair length or range of lengths a separate lossy counter
  can be specified, for example: 1:0.5:1.5 2-3:1.0:2.0 4-7:3.0:5.0
 \item \verb|--lambda| - the value for lambda parameter, real number from
  (0.0, 1.0) interval. Default value is 0.5.
\end{itemize}

Options related to phrase pairs scoring:
\begin{itemize}
 \item \verb|--phrase-table-file| - path to phrase table file.
  The inverse file has $.inv$ extension attached to its name automatically.
 \item \verb|--dir-lex-table| - path to direct lexical table.
  If given, direct lexical score $lex(e|f)$ will be included in phrase table.
  May be passed also as 4-th positional parameter.
 \item \verb|--inv-lex-table| - path to inverse lexical table.  
  If given, inverse lexical score $lex(f|e)$ will be included in phrase table.
  May be passed also as 5-th positional parameter.
 \item \verb|--OnlyDirect| - print only direct scores $p(e|f)$ and $lex(e|f)$
 \item \verb|--NoLex| - do not include lexical scores.
  It is sufficient to omit \verb|--direct-lex-table| and \verb|--inverse-lex-table|
  options to not include lexical scores, but this switch can be used to
  override their presence.
 \item \verb|--NoPhraseCount| - do not include phrase counts.  
 \item \verb|--WordAlignment| - print word alignment.
 % TODO: Finish.
\end{itemize}

Options related to both phrase pairs extraction and scoring:
\begin{itemize}
 \item \verb|--GZOutput| - gzip the output (automatically
  adds $.gz$ extension to the filenames provided).
\end{itemize}

Finally, three program options that provides some information about the program:
\begin{itemize}
 \item \verb|--help| - prints the program help.
 \item \verb|--info| - prints basic information about the program.
 \item \verb|--version| - prints program version.
\end{itemize}

\subsection{Missing features}

In the moment eppex cannot extract orientation info data required for reordering models training.

Also some of the recently added features to the core scoring tool are yet missing in \eppex{},
namely:
