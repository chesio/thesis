\chapter{Installation}
\label{chap:installation}

% TODO: Pull-in eppex to the upstream Moses repository?
% \emph{Eppex} is shipped along with the Moses system and can be obtained the
% same way: by checking out Moses repository at Github.

\section*{Prerequisities}

Only Boost is required to compile and run \eppex{}.
If you had successfully compiled Moses then you probably already have all you need in place.

Just in case you did not want to install Boost as a whole, but only the necessary
parts of it, you will need to get two libraries (build them or install
them via your packaging system): \emph{Program Options} and \emph{IOStreams}.
\Eppex{} also requires working include path access to following
header-only libraries: \emph{Integer}, \emph{Iterator}, \emph{Pool} and \emph{Tokenizer}.

Besides that you will probably like to employ the power of hash tables
that comes with the recent version of C++ standard.\footurl{http://en.cppreference.com/w/cpp/container/unordered_set}
Most of decent C++ compilers will have this feature available,
although some may require to explicitly request compilation with C++11
features.\footnote{For example GCC requires the command-line parameter
\texttt{-std=c++0x} to be passed when compiling (or \texttt{-std=c++11}
with GCC 4.7 and later).}

\section*{Download}

\emph{Eppex} is hosted on Github\footurl{https://www.github.com} in
a fork\footurl{https://github.com/chesio/mosesdecoder}
of core Moses repository\footurl{https://github.com/moses-smt/mosesdecoder}.
If you have your own local clone of Moses repo then all you need is to add
the fork as a new remote, fetch from it and checkout the remote branch
\texttt{eppex} into some new local branch.
\begin{verbatim}
 cd <your-mosesdecoder-dir>
 git remote add eppex https://github.com/chesio/mosesdecoder.git
 git fetch eppex
 git checkout -b eppex eppex/eppex
\end{verbatim}

Important note: \eppex{} branch is based on \emph{RELEASE-1.0},
not the current master. Should you like to use \eppex{} with the most
recent Moses code, you may try merging branch \texttt{master} into \texttt{eppex}:
\begin{verbatim}
 git merge master
\end{verbatim}

Be warned that you may encounter some merge conflicts in \emph{train-model.perl}
as this script gets frequently updated.
In such a case feel free to contact the author of \eppex{},
he will be happy to help you.

\section*{Installation}

\Eppex{} related files may be found in \texttt{contrib/eppex} directory of \eppex{} fork
and for the time being the executable has to be compiled separately from the Moses
build.\footnote{For the build instructions on Moses consult the file \texttt{BUILD-INSTRUCTIONS.txt}.}
Some implementation features and functionality of \eppex{} is determined during compile-time
and you are encouraged to alter the defaults to suit your needs and your compiler capabilities.
The specific way of how to alter them depends on selected compilation method (\emph{Boost.Build} or
\emph{make} -- see below), the default behavior when no changes are made is following:
\begin{itemize}
  \item \eppex{} uses \verb|std::map| and \verb|std::set| as associative containers
  \item \eppex{} allows to extract phrase pairs of maximum length limit up to 8
\end{itemize}

Once compiled you will find \eppex{} binary in the \texttt{contrib/eppex} directory.
You are free to move it to whatever location suits your working environment,
just do not forget to pass the full path to \emph{train-model.perl} invocation.

\subsection*{Installation using Boost.Build}

This approach is recommended since \emph{Boost.Build}\footurl{http://www.boost.org/boost-build2/doc/html/index.html}
is required by Moses install process as well and if you do not have a working
\emph{Boost.Build} setup on your machine, you may try to use the one shipped with Moses:
just run \emph{bjam} shell script from Moses root instead of the system \emph{bjam}.\footnote{Note
that the first invocation will attempt to compile the actual \emph{bjam} executable from sources.}

If you would like to compile \eppex{} with faster (unordered) associative containers
just pass \texttt{--with-hashtables} flag to \emph{bjam} command.
The limit for maximum phrase length can be risen to 128 by adding
\texttt{--allow-long-phrases}.
\begin{verbatim}
  cd contrib/eppex
  bjam [--with-hashtables] [--allow-long-phrases]
  # or using Moses bjam: ../../bjam <options>
\end{verbatim}

\subsection*{Installation using make}
The \texttt{contrib/eppex} directory also contains a Makefile that may be used to compile
\eppex{} and should work on most Linux systems that have GCC compiler installed.
The faster associative containers and higher maximum phrase length limit can be activated
via environment variables:
\begin{verbatim}
  cd contrib/eppex
  [USE_HASHTABLES=1] [ALLOW_LONG_PHRASES=1] make all
\end{verbatim}
