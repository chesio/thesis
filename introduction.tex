\chapter{Introduction}
\label{chap:introduction}

\section{Thesis outline}

We start with introduction of the phrase tables concept,
portray their role in phrase-based SMT systems and describe
methods of phrase tables construction, filtering and compression that
are available in Moses. All of this is part of Chapter 2.

In Chapter 3 we introduce the algorithm that is the basis of our implementation
of on-the-fly filtration and show the properties of the output produced
by the algorithm that make it particularly applicable in the filtration process.

Chapter 4 is devoted to in-depth description of implementation details of our
phrase table extraction tool, \emph{epochal extractor} (or shortly \eppex{}).
Notably, various memory-management optimizations are mentioned.
Also all the program options are explained with examples of their usage.

To assess \eppex{} usability in real world applications, we carried out a set
of carefully crafted experiments aiming at comparison of resources usage as well as
the ultimate translation quality of \eppex{} and some of the methods mentioned
in Chapter 2.
Detailed design of experiments is subject of Chapter 5,
while the results are discussed in Chapter 6.

Chapter 7 is used as a storage for temporary notes, random thoughts and
useful LaTeX commands. It is going to be removed from final version
of this work, along with this note, so if you are reading this, you have
a draft version only or the work never reached its final state.

In last chapter we comment on our results and provide a conclusion of
what have been done and what can be done in the future work on this topic.

In Appendix A you may find \eppex{} installation instructions.
