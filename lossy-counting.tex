\chapter{Lossy Counting algorithm}
\label{chap:lossy-counting}

\setlength{\epigraphwidth}{1.0\textwidth}
\epigraph{(...) since frequent source phrases are more useful than the infrequent ones.}{--- R. Zens et al., A Systematic Comparison of Phrase Table Pruning Techniques}

The \emph{Lossy Counting (LC) algorithm} \citep{manku:lossycounting} is
a deterministic algorithm that computes approximate frequency counts
over virtually infinite stream of input items.
Although the counts are approximate, the error is guaranteed not to exceed
a user-specified parameter.

The algorithm has been initially developed to help with identification of
large network flows that is essential to network monitoring, but poses a big
computational problem due to the variability of monitored packets and limited
memory of networking hardware.
The specific design of the algorithm results in regular pruning of low frequent
items from the frequency counts list, as the algorithm runs, and thus maintains
this list small-sized, but in the same time ensures that high frequent items are
retained (and identified).

Many statistically-driven methods of natural language processing internally
employ the same kind of process: they gather frequency counts over a set of highly
variable items in order to estimate the properties of statistical model they are
based on.
Thus, the idea to use Lossy Counting algorithm in the field of NLP is not new:
\citet{goyal:streaminglm} used approximate n-gram frequency counts to build language
models from billion-word monolingual corpora using a conventional desktop machine and
applied these models successfully in SMT system achieving no significant loss in BLEU
score. We shall note that their work particularly inspired this thesis.

\section{Definition}
\label{sec:lossy-counting-definition}

At its initialization, Lossy Counting algorithm expects to be given two parameters:
\emph{support} $s \in (0,1)$ and \emph{error} $\epsilon \in (0,1)$, such
that $\epsilon \ll s$.
At any point of time, after being fed with $N$ items from the stream,
the algorithm can be asked to output the list of items with their approximate
frequencies that satisfies the following guarantees:
\begin{itemize}
 \item All items whose true frequency is greater or equal to $sN$ are output
  (\emph{no false negatives}).\footnote{\citet{manku:lossycounting} formulates this
    guarantee using term "exceeds", but from the definition of the algorithm and
    the supportive lemmas it is clear that the guarantee can be safely formulated with
    "is greater or equal to" assertion -- proof is given in \Sref{sec:positive-limit-validity}.}
 \item No item whose true frequency is less than $(s - \epsilon)N$ is output
  (\emph{few false positives}).
 \item Estimated frequencies are less than the true frequencies by at most
  $\epsilon N$ (\emph{close-to-exact frequencies}).
 \item The space used by the algorithm is $O(\frac{1}{\epsilon}~log \epsilon N)$.
\end{itemize}

Conceptually, Lossy Counting algorithm divides the incoming stream of items
into epochs\footnote{In the original paper the term "buckets" is used.}
of fixed size $w = \lceil \frac{1}{\epsilon} \rceil$
(thus the name \emph{epochal extraction}).
Epochs are numbered with \emph{IDs}, starting from 1.
Given the number of currently processed items, $N > 0$, we may denote
\emph{current epoch ID} as $I = \lceil \frac{N}{w} \rceil$.

Internally, the algorithm maintains a data structure $D$ consisting of triples
$(e, f, \Delta)$, where $e$ is an element from the stream, $f$ is its estimated
frequency and $\Delta$ is the maximum possible error in $f$.
Initially, $D$ is empty.
When a new item $e$ arrives, a lookup for $e$ in $D$ is performed.
If $e$ is already present, its frequency $f$ is incremented by one.
Otherwise a new triple ($e$, 1, $I-1$) is added to $D$,
where $I$ denotes the ID of current epoch as defined above.

At the end of each epoch (determined by $N \equiv 0~mod~w$), the algorithm
prunes off all items whose maximum true frequency is small.
Formally, at the end of the epoch $I$, all triples satisfying the condition
$f + \Delta \leq I$ are removed from $D$.
When all elements in the stream have been processed, the algorithm returns all
triples ($e$, $f$, $\Delta$) where $f \geq (s-\epsilon)N$.

For an entry $(e, f, \Delta)$, the value of $f$ represents the exact frequency
since this entry was inserted into $D$.
The value of $\Delta$ assigned to a new entry is the maximum number of times
$e$ could have occurred in the first $I-1$ buckets. This is exactly $I-1$.
This value remains unchanged as long as the entry remains in $D$.

The intuitive idea behind the Lossy Counting algorithm is that frequent elements
show up more than once within each epoch so their frequencies are increased enough
to survive the filtering.

\section{Applicability in phrase table pruning}

% LC in the perspective of pruning methods
From the perspective of phrase table pruning techniques discussed in
\Sref{sec:phrase-table-pruning}, construction of phrase table based on frequency counts
collected via Lossy Counting algorithm has a similar effect to count-based pruning of
phrase table obtained by a usual extraction process:
specifically, the assertion, that no item whose true frequency is less than
$(s - \epsilon)N$ will be output, is functionally equal to the criterion of
count-based pruning formulated in \Eref{eq:count-based-pruning}.

% From initialization with thresholds to initialization with limits.
The only issue with Lossy Counting algorithm being applied in the similar manner as
count-based pruning is the necessity to initialize the algorithm with such a values of
\emph{support threshold} $s$ and \emph{error threshold} $\epsilon$ that will ensure
the effective pruning threshold to be equal to a desired value of count-based
motivated threshold $\theta_{c}$.
In other words, for the desired value of $\theta_{c}$ the equation $\theta_{c} = (s - \epsilon)N$
must hold in the moment the algorithm is asked to dump the output (and after it read $N$ items).
This is obviously not possible without the initial knowledge of the number of read
items $N$ and thus either a predefined limit must be set in advance (lesser than
the actual number of items in the stream) or a quick reading loop over the whole stream
must be done.

However, when the number of items that will be read from the input stream is known
in advance, it is possible to propose some modifications to the means of algorithm
initialization.
Instead of requiring the user to come up with \emph{support} and \emph{error} thresholds,
we may offer a slightly more intuitive parameters that we call \emph{limits} for more clarity:
\begin{itemize}
  \item A \emph{negative limit} $n$ - a positive integer value such that no item
    whose true frequency is equal or less than $n$ will be output.
  \item A \emph{positive limit} $p$ - a positive integer value such that all items
    whose true frequency is equal or greater than $p$ will be output.
\end{itemize}

The actual values of \emph{support} $s$ and \emph{error} $\epsilon$ thresholds have then
to be carefully derived from the definition of \emph{limits} to ensure their soundness.
The knowledge of input stream size is crucial here:
\begin{itemize}
  \item \emph{Support} threshold $s$ is instantiated as $s = p / N$.
    This way, we get $p = sN$, and since Lossy Counting guarantees that all items whose
    true frequency is greater or equal to $sN$ will be output, the definition of
    \emph{positive limit} is sound.
  \item \emph{Error} threshold $\epsilon$ is instantiated as $\epsilon = (p - n - \lambda) / N$,
    where $\lambda \in (0,1)$.
    By reformulating the equation for $n$ and applying the fact that $p = sN$,
    we get $n = sN - \epsilon N - \lambda = (s - \epsilon)N - \lambda$.
    This way $n < (s - \epsilon)N$ and since Lossy Counting guarantees that no item whose
    true frequency is less than $(s - \epsilon)N$ is output, the same applies for
    \emph{negative limit} $n$ and thus its definition is also sound.
    Moreover, by limiting $\lambda < 1$ we ensured that $n$ is the largest integer with such
    property, that is: $\lfloor (s - \epsilon)N \rfloor = n$.
\end{itemize}

The definition of \emph{positive and negative limits} trivially enforces that $n < p$ and
one theoretically interesting setting of \emph{limits} is $p = n + 1$.
In such a case Lossy Counting driven phrase table extraction gives exactly the same results
as count-based pruning the table with $\theta_{c} = p$.
Consequently, the algorithm must return exact frequency counts and indeed,
the maximum error of estimation expressed by $\lfloor \epsilon N\rfloor $ evaluates to
$\lfloor 1 - \lambda \rfloor = 0$ for $\lambda \in (0,1)$.
However, since no estimation error is made, the space requirements of algorithm
reach the upper bound for a given input size.
A more space-conscious settings will therefore employ limits $p$ and $n$,
such that $p > n + 1$, allowing for nonzero estimation error.

\section{A stronger no false negatives claim}
\label{sec:positive-limit-validity}

The original Lossy Counting guarantee about \emph{no false negatives} states that all items
whose true frequency \emph{exceeds} $sN$ are output \citep[p. 3]{manku:lossycounting}, but
for the purpose of \emph{positive limit} definition we reformulated this guarantee using
a stronger claim that all items whose true frequency \emph{is greater or equal to} $sN$ are
output. Or more formally, we claimed that any item $e$ with true frequency $f_{e} \geq sN$
will be kept in the output of Lossy Counting algorithm instantiated with
\emph{support threshold} $s$ and \emph{error threshold} $\epsilon$
(we reuse the notation from \Sref{sec:lossy-counting-definition} in the rest of this section).

We are now going to prove this stronger claim using just the properties of Lossy Counting
algorithm and two supportive lemmas from the original paper:
\begin{enumerate}
  \item If $e$ does not appear in $D$, then $f_{e} \leq \epsilon N$.
  \item If $(e,f,\Delta) \in D$, then $f \leq f_{e} \leq f + \epsilon N$.
\end{enumerate}

An item $e$ can be pruned off by Lossy Counting algorithm either at the end of an epoch or
during the final pruning that is executed when the output is requested.

As for the first case, should the item $e$ be not present in $D$ in the moment of final
pruning as a result of an earlier end-of-epoch pruning, the first supportive lemma states
that its true frequency $f_e \leq \epsilon N$, but since the initialization of the algorithm
enforces $s > \epsilon$, for any item $e$ with true frequency $f_{e} \geq sN$ it trivially
holds that $f_{e} > \epsilon N$ and thus it is proved by contradiction that any such item
must have record in $D$.

As for the second case, the final pruning criterion removes all items with estimated frequency
$f < (s - \epsilon)N$. From the second lemma we know that estimated frequency $f$ is always
equal or less than true frequency $f_{e}$, but even in the worst case scenario (the biggest
estimation error possible) $f \geq f_{e} - \epsilon N$.
Thus, for a phrase pair to be pruned it must hold that $f_{e} - \epsilon N \leq f < (s - \epsilon) N$.
Ruling out $f$ and adding $\epsilon N$ to both sides of inequation simplifies it to $f_{e} < sN$,
making it obvious that any item $e$ with $f_{e} \geq sN$ cannot be removed by the final pruning.
