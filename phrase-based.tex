% In this chapter:
% - introduction of phrase based SMT
% - description of phrase table
% - description of phrase table creation in Moses
% -- with alternatives: memscore
% - phrase table post filtering
% -- sigfilter
% - alternatives to precomputed phrase table
% -- dynamic suffix arrays
% - mention phrase table compression tools

\chapter{Phrase tables in phrase-based Statistical Machine Translation}
\label{chap:phrase-based}

% Briefly introduce the concept of phrase-based SMT
% (SMT book, Chapter 5)

Phrase-based models are nowadays considered the most eminent approach in the field of Statistical Machine Translation.
Their design is very simple, yet powerful enough to compete with linguistically motivated models and
they conceptually benefit from the increasing amount of training data and computational resources available.

\section{Phrase-based Statistical Machine Translation}

Phrase-based models are conceptually based on idea of straight-forward word-for-word translation,
but instead of using only single word as translation unit they allow for sequence of words -- a phrase
-- to be translated at once.
By using phrases rather than words the model is implicitly able to map non-regular translation patterns
that are problematic in simple word-for-word translation.
One example of such pattern is word in source language that translates to multiple words in target language.
Another example is reordering of translated part when source and
target language differ in word order semantics
-- this constitutes a big problem even with phrase-based models,
but local reordering may be captured more easily within the phrases.

\subsection{Mathematical Definition}
% TODO.


\section{Phrase translation table}
% What's the phrase table?

\emph{Phrase translation table} in SMT systems captures the mapping
between phrases in source language and phrases in target language
along with scores that should reflect the quality of translating a particular
source language phrase as a particular target language phrase.
The concept of phrase table is not related to any linguistic notion,
neither the particular phrase pair nor the scores have to be meaningful
in terms of syntax or semantics. The current phrase-based systems construct
phrase tables using \emph{maximum likelihood} estimates calculated from (almost)
plain text input data (hence statistical). The utility of translating phrase
$s$ into phrase $t$ is then represented by maximum likelihood probability of
the phrase $t$ given the phrase $s$.

An example of a simple phrase table is presented by \Tref{phrase-table-example}.
The table contains various options of how to translate Czech phrase
"pes" (dog) into English.

% TODO: Include real phrase table example.
% TODO: Include a phrase, not a single word only.
\begin{table}[h]
\centering
\begin{tabular}{ l l l}
Czech (s) & English (t) & Probability p(t|s) \\
\hline
\hline
pes & dog & 0.8 \\
pes & cat & 0.1 \\
pes & wolf & 0.07 \\
pes & hamster & 0.03 \\
\hline
\hline
\end{tabular}
\caption{\label{phrase-table-example}Phrase translation table extract.}
\end{table}

We may note that the example phrase table contains some erroneous items,
as dog is unlikely to become cat just by switching to a different language.
The fact is that standard methods of phrase table construction may produce such
unlikely phrase translations, one reason being that their input data usually
contains a lot of noise.
We will discuss this problem further in the following sections.

% TODO: Ctrl+C Ctrl+V from the eppex paper.

Phrase tables in Statistical Machine Translation (SMT) systems generally take
the form of a list of pairs of phrases $s$ and $t$, $s$ being the phrase from
the source language and $t$ being the phrase from the target language, along
with scores that should reflect the goodness of translating $s$ as $t$.
The standard approach to obtain such scores is to use \emph{maximum likelihood
probability} of the phrase $t$ given the phrase $s$ and vice versa.
The probabilities $p(s|t)$ and $p(t|s)$ are often referred to as
\emph{forward} and \emph{reverse} \emph{translation probabilities}.


\subsection{Phrase table creation}

To estimate $p(s|t)$ and $p(t|s)$, frequency counts $C(t,s)$, $C(s)$ and
$C(t)$ are usually collected from the entire training corpus.
For substantial coverage of source and target languages, such corpora are
often very big so all phrase pairs and their counts cannot fit in the
physical memory of the computer.
To overcome this limitation, phrase table construction methods often simply
dump observed phrases to local disk and sort and count them on disk.
This approach allows to construct phrase tables of size limited only by the
capacity of the disk.
The obvious drawback of this solution is that much more time is needed
to build the table.

% What's the state-of-art implementation of phrase table extraction in Moses?
% Introduce train-model.perl and steps of training pipeline.

Moses comes with a training script that incorporates all the steps required
for creation of ready-to-go translation system from the parallel corpus.

\subsection{Phrase table pruning}

% Introduce Johnson's significance filtering.

\subsection{Phrase table compression}

% Introduce phrase table compacting tool by Marcin.
