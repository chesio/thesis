\chapter{Results}
\label{chap:results}

\section{Cs-En dataset}
\label{sec:cs-en-results}

We perform 13 experimental phrase table extractions for Cs-En dataset,
the comprehensive list is presented by \Tref{cs-en-wmt13-scenarios}.
We included one more optimized baseline with 8 cores, because the difference in
wall clock time necessary to create phrase table between runs with 4 and 8 cores
was considerable.
Sorting processes in both optimized baselines were given 18~GB of memory, so the virtual
peak of entire pipeline reached approximately the same value as in case of \eppex{}
run with no pruning.

% Cs-En: description and parameters of experiments
\begin{table}[ht]
\centering
\begin{tabular}{ r p{10cm} }
name & description and parameters \\
\hline
\hline
def-base        & Standard Moses pipeline with no special parameters \\
multi-base      & Standard Moses pipeline with \verb|--cores 4| \\
comp-base       & Standard Moses pipeline with \verb|--sort-compress gzip| and \verb|--cores 4| \\
opt-base        & Standard Moses pipeline with \verb|--sort-buffer 18G|, \verb|--sort-compress gzip| and \verb|--cores 4| \\
opt-c8-base     & Standard Moses pipeline with \verb|--sort-buffer 18G|, \verb|--sort-compress gzip| and \verb|--cores 8| \\
\hline
eppex zero      & \eppex{} set to no pruning and \verb|--GZOutput| option \\
eppex def.      & \eppex{} with \verb|--limits| set to \verb|1-3:0:1,4-5:1:4,6-7:4:8| and \verb|--GZOutput| option \\
eppex zero-n    & \eppex{} with \verb|--limits| set to \verb|1:0:1,2:0:2,...,7:0:7| and \verb|--GZOutput| option \\
eppex 1:n+1     & \eppex{} with \verb|--limits| set to \verb|1:1:2,2:1:3,...,7:1:8| and \verb|--GZOutput| option \\
\hline
sigfilter a-e   & baseline followed by significance filtering with pruning threshold $\alpha - \epsilon$ \\
sigfilter a+e   & baseline followed by significance filtering with pruning threshold $\alpha + \epsilon$ \\
sigfilter 30 a+e  & baseline followed by significance filtering with cutoff limit of 30 and pruning threshold $\alpha + \epsilon$ \\
sigfilter 30    & baseline followed by significance filtering with cutoff limit of 30 \\
\hline
\hline
\end{tabular}
\caption{\label{cs-en-wmt13-scenarios}
List of various experiments and their settings for "Cs-En" setup.}
\end{table}

\subsection{Translation phrase table size and quality}

\Tref{cs-en-wmt13-pt-size-and-bleu} presents phrase table sizes and BLEU scores for all
distinct phrase tables created in our experiments.
Despite that phrase table sizes are very different, with the smallest phrase table being only
$1/13$ size of the biggest one, the achieved BLEU scores are very tight and the difference
between the best and worst is only 0.61 point.

% Cs-En: phrase tables sizes and BLEU scores
\begin{table}[ht]
\centering
\begin{tabular}{ | c | c c | c c | }
\hline
 & \multicolumn{2}{|c|}{final phrase table size} & \multicolumn{2}{|c|}{BLEU score} \\
experiment & phrase pairs & .gz file size & wmt-12 & wmt-13 \\
\hline
\hline
baseline          & 336.0~M & 8.8~GB & 0.2327 & 0.2583 \\
sigfilter 30      & 301.9~M & 8.2~GB & 0.2312 & 0.2562 \\
sigfilter a-e     & 203.1~M & 5.9~GB & 0.2301 & 0.2560 \\
eppex def.        & 109.8~M & 2.7~GB & \textbf{0.2338} & \textbf{0.2598} \\
eppex zero-n      &  86.2~M & 2.3~GB & 0.2324 & 0.2586 \\
sigfilter a+e     &  70.0~M & 1.9~GB & 0.2321 & 0.2559 \\
sigfilter 30 a+e  &  60.2~M & 1.7~GB & 0.2305 & 0.2563 \\
eppex 1:n+1       &  25.7~M & 0.7~GB & 0.2272 & 0.2537 \\
\hline
\end{tabular}
\caption{\label{cs-en-wmt13-pt-size-and-bleu}
Phrase table sizes and BLEU scores for various experiments of "Cs-En" setup.}
\end{table}

The \emph{eppex defensive} experiment based on \emph{eppex 1-in} scenario from 2011 experiments
again proved to be very competitive: the phrase table has only $1/3$ of size of the baseline,
but the BLEU score is actually better (although not significantly).

\subsection{Memory and time requirements}

\Tref{cs-en-wmt13-time-benchmarks} presents the amount of time necessary to finish
phrase table extraction with various systems and their configurations.

% Cs-En: baseline and eppex wall clock and CPU time values
\begin{table}[ht]
\centering
\begin{tabular}{ | c | c c | c c | c c | }
\hline
 & \multicolumn{2}{|c|}{total time} & \multicolumn{2}{|c|}{phrase extraction} & \multicolumn{2}{|c|}{phrase scoring} \\
experiment & wall & CPU & wall & CPU & wall & CPU \\
\hline
\hline
def-base      & 15:36:27 & 15:27:09 & 07:07:37 & 06:27:59 & 08:28:49 & 08:59:09 \\
multi-base    & 10:25:59 & 19:25:36 & 04:54:29 & 06:40:45 & 05:31:29 & 12:44:51 \\
comp-base     & 10:33:45 & 25:25:06 & 04:08:42 & 11:19:53 & 06:25:02 & 14:05:12 \\
opt-base      & 08:16:45 & 19:54:35 & 02:38:39 & 06:57:33 & 05:38:05 & 12:57:02 \\
opt-c8-base   & 07:01:39 & 20:44:59 & 02:20:57 & 07:10:41 & 04:40:42 & 13:34:17 \\
eppex zero    & 02:56:57 & 02:56:35 &       -- &       -- &       -- &       -- \\
\hline
eppex def.    & 01:35:25 & 01:35:14 &       -- &       -- &       -- &       -- \\
eppex zero-n  & 01:31:28 & 01:31:17 &       -- &       -- &       -- &       -- \\
eppex 1:n+1   & 01:13:53 & 01:13:43 &       -- &       -- &       -- &       -- \\
\hline
\end{tabular}
\caption{\label{cs-en-wmt13-time-benchmarks}
Wallclock times and CPU usage values (in hh:mm:ss format) of the phrase table
construction for various experiments of "Cs-En" setup.}
\end{table}

Even with no pruning \eppex{} is capable of doing the phrase table construction more
than twice as fast as the most challenging baseline (and using 7 times less CPU time).
When compared to default baseline the difference is even more pronounced with \eppex{}
being 5~times as fast.

From the comparison of default and multi-core baseline it seems clear that whenever there
is a possibility to employ multiple cores it should be taken: just by running with 4 cores
the baseline execution time has been cut to $2/3$.
Doubling number of cores from 4 to 8 in optimized experiments resulted in further reduction
of execution time, although not as significant (by 15\%).

An important observation is that adding only the option to make \texttt{sort} program gzip
its temporary data, did not help to decrease the total execution time.
More precisely, it did help in phrase extraction step ($-46~mins$), but did the very opposite
in scoring step ($+54~mins$), so the overall impact on the total wall clock time was eventually
a bit negative ($+8~mins$).
A reasonable explanation for such, perhaps unexpected, behavior stems from the fact that in phrase
extraction step both phrase table halves are sorted simultaneously, whereas in scoring step
only indirect half is sorted:
apparently, in our environment forcing \texttt{sort} to gzip its temporary data makes it run
slower when there are no other processes accessing the disk (as in scoring step), but reducing
the disk access by the same means when two similar sort tasks run simultaneously can benefit
in both of them finishing faster (as in phrase extraction step).
This explanation might be further supported by the fact that a lot of \texttt{sort} temporary
data gzipping took place in \emph{comp-base} experiment, because the striking increase of CPU time
consumption can only be attributed to this extra gzipping:
phrase extraction required almost 4.5 CPU hours more, scoring required almost 1.5 CPU hours
more.\footnote{Given that there are two and one sorting processes, one could expect the ratio
4.5:1.5 of additional CPU time to be closer to 2:1, but after phrase extraction phrase table halves
contain every phrase pair occurrence, whereas after scoring they contain only unique phrase pairs,
so the size of data to sort is rather close to 3:1 than 2:1 ratio.}
The similar conclusions might be made also by examination of partial wall clock times of \emph{multi-base}
and \emph{opt-base} experiments: while phrase extraction time felt down from 4h~54mins to 2h~39mins
(ie. almost to $1/2$), the scoring times jumped up a bit from 5h~31mins to 5h~38mins.
Thus, even the significantly increased sorting buffer managed only to neutralize the negative impact of
temporary data compression on wall clock time of scoring step, but even more pronounced the positive
impact of the same settings on phrase extraction step.

However, to put things in wider perspective, we believe that the option to compress temporary data of
\texttt{sort} routine has been introduced in order to limit the peak usage of disk space, which may
otherwise pose a limiting factor in some environments, as the figures presented in the following text
demonstrate.

\Tref{cs-en-wmt13-vm-and-disk-usage-peaks} illustrates both memory and disk space demands of baseline
and \eppex{} experiments.

% Cs-En: virtual memory and disk usage peaks
\begin{table}[ht]
\centering
\begin{tabular}{ | c | r l | c | }
\hline
 & \multicolumn{2}{|c|}{VM peak} & \\
experiment & size & step & disk usage peak \\
\hline
\hline
def-base       &  2.0~GB &    scoring & 206.7~GB \\
multi-base     &  6.2~GB &    scoring & 213.5~GB \\
comp-base      &  6.2~GB &    scoring &  43.4~GB \\
opt-base       & 36.2~GB & extraction &  42.8~GB \\
opt-c8-base    & 36.2~GB & extraction &  42.8~GB \\
eppex zero     & 36.8~GB &      eppex &   8.8~GB \\
\hline
eppex def.     & 18.1~GB &      eppex &   2.7~GB \\
eppex zero-n   & 10.5~GB &      eppex &   2.2~GB \\
eppex 1:n+1    &  9.6~GB &      eppex &   0.7~GB \\
\hline
\end{tabular}
\caption{\label{cs-en-wmt13-vm-and-disk-usage-peaks}
Virtual memory peaks of the phrase table construction for various experiments of "Cs-En" setup.}
\end{table}

\Eppex{} memory demands are largely dependent on the amount of pruning requested: with \emph{zero}
pruning almost 37~GB of memory was consumed, \emph{defensive} pruning required only half as much
memory and harsh \emph{1:n+1} pruning only $1/4$ (but still almost 10~GB).
As noted before, \eppex{} produces no temporary data, therefore disk usage peak is low and strictly
reflects the size of the produced phrase table.

The default baseline experiment had the lowest memory demands from all experiments, it needed only 2~GB
of memory to finish.
With non-default settings the memory consumption tripled when multiple cores were utilized (in \emph{multi-base}
and \emph{comp-base} experiments), except for cases when it was dominated by the size of buffer dedicated
to \texttt{sort} routine (in optimized experiments).
The disk usage peak exceeded 200~GB in experiments without compression of \texttt{sort} temporary data,
with the compression it lowered to approximately 40~GB (notably still above the memory peak of \eppex{}
with no pruning).

Finally, \Tref{cs-en-wmt13-sigfilter-runtime-benchmarks} presents the runtime benchmarking
values for different runs of significance filtering.
% TODO: Any comment?

% Cs-En: sigfilter runtime requirements
\begin{table}[ht]
\centering
\begin{tabular}{ | c | c c | c c | }
\hline
 & \multicolumn{2}{|c|}{time} & \multicolumn{2}{|c|}{VM peak} \\
experiment & wall & CPU & size & step \\
\hline
\hline
sigfilter a-e     & 10:09:39 & 11:18:56 & 6.7~GB & filtering \\
sigfilter a+e     & 09:52:37 & 10:35:03 & 6.8~GB & filtering \\
sigfilter 30 a+e  & 08:04:25 & 08:44:39 & 6.7~GB & filtering \\
sigfilter 30      & 02:32:37 & 03:22:35 & 2.7~GB & indexing target \\
\hline
\end{tabular}
\caption{\label{cs-en-wmt13-sigfilter-runtime-benchmarks}
Wallclock times and CPU usage values (in hh:mm:ss format), virtual memory peak and the step, in which
it occurred of significance filtering for various settings of \emph{sigfilter} in "Cs-En" setup.}
\end{table}

\subsection{Eppex and memory demands}

% TODO: Attach a graph displaying the rate of growth of memory usage with 1M steps

We performed an additional series of experimental phrase table extractions
to get an insight on how the memory usage of \eppex{} relates to the amount of
phrase pairs extracted from the parallel corpus.

...

\section{The big Fr-En dataset}
\label{sec:fr-en-results}

% TODO: Run and update.
\Tref{fr-en-80-scenarios} presents the draft of intended experiments with "Fr-En" dataset.

% Fr-En: description and parameters of experiments
\begin{table}[ht]
\centering
\begin{tabular}{ r p{10cm} }
name & description and parameters \\
\hline
\hline
opt-base      & Standard Moses pipeline with \verb|--sort-buffer 100G|, \verb|--sort-compress gzip| and \verb|--cores 4| \\
opt-c8-base   & Standard Moses pipeline with \verb|--sort-buffer 100G|, \verb|--sort-compress gzip| and \verb|--cores 8| \\
eppex zero    & \eppex{} set to no pruning and \verb|--GZOutput| option \\
eppex def.    & \eppex{} with \verb|--limits| set to \verb|1-3:0:1,4-5:1:4,6-7:4:8| and \verb|--GZOutput| option \\
eppex zero-n  & \eppex{} with \verb|--limits| set to \verb|1:0:1,2:0:2,...,7:0:7| and \verb|--GZOutput| option \\
eppex 1:n+1   & \eppex{} with \verb|--limits| set to \verb|1:1:2,2:1:3,...,7:1:8| and \verb|--GZOutput| option \\
eppex 2:n+2   & \eppex{} with \verb|--limits| set to \verb|1:2:3,2:2:4,...,7:2:9| and \verb|--GZOutput| option \\
\hline
\hline
\end{tabular}
\caption{\label{fr-en-80-scenarios}
List of various experiments and their settings for "Fr-En" setup.}
\end{table}

\subsection{Translation phrase table size and quality}

\Tref{fr-en-pt-size-and-bleu} presents phrase table sizes and BLEU scores for all
distinct phrase tables created in our experiments.

% Fr-En: phrase tables sizes and BLEU scores
\begin{table}[ht]
\centering
\begin{tabular}{ | c | c c | c | }
\hline
 & \multicolumn{2}{|c|}{final phrase table size} & \\
experiment & phrase pairs & .gz file size & BLEU score \\
\hline
\hline
baseline          & 1782.0~M & 42.5~GB & ? \\ % TODO.
sigfilter 30      &    ?.?~M &  ?.?~GB & ? \\ % TODO.
sigfilter a-e     &    ?.?~M &  ?.?~GB & ? \\ % TODO.
eppex def.        &  463.4~M & 10.6~GB & ? \\ % TODO.
eppex zero-n      &  283.2~M &  7.0~GB & ? \\ % TODO.
sigfilter a+e     &    ?.?~M &  ?.?~GB & ? \\ % TODO.
sigfilter 30 a+e  &    ?.?~M &  ?.?~GB & ? \\ % TODO.
eppex 1:n+1       &  127.9~M &  3.2~GB & ? \\ % TODO.
eppex 2:n+2       &   77.0~M &  2.0~GB & ? \\ % TODO.
\hline
\end{tabular}
\caption{\label{fr-en-pt-size-and-bleu}
Phrase table sizes and BLEU scores for various experiments of "Fr-En" setup.}
\end{table}

...

\subsection{Memory and time requirements}

\Tref{fr-en-time-benchmarks} presents the amount of time necessary to finish
phrase table extraction with various systems and their configurations.

% Fr-En: baseline and eppex wall clock and CPU time values
\begin{table}[ht]
\centering
\begin{tabular}{ | c | c c | c c | c c | }
\hline
 & \multicolumn{2}{|c|}{total time} & \multicolumn{2}{|c|}{phrase extraction} & \multicolumn{2}{|c|}{phrase scoring} \\
experiment & wall & CPU & wall & CPU & wall & CPU \\
\hline
\hline
opt-base      & 56:45:41 & 146:06:57 & 14:59:56 & 39:12:55 & 41:45:45 & 106:54:01 \\
opt-c8-base   & 45:05:05 & 149:48:59 & 12:43:22 & 39:41:57 & 32:21:42 & 110:07:02 \\
eppex zero    & 25:28:49 &  25:24:43 &       -- &       -- &       -- &        -- \\
\hline
eppex def.    & 12:14:54 &  12:13:28 &       -- &       -- &       -- &        -- \\
eppex zero-n  & 10:40:26 &  10:39:12 &       -- &       -- &       -- &        -- \\
eppex 1:n+1   & 09:29:50 &  09:28:40 &       -- &       -- &       -- &        -- \\
eppex 2:n+2   & 09:18:18 &  09:13:48 &       -- &       -- &       -- &        -- \\
\hline
\end{tabular}
\caption{\label{fr-en-time-benchmarks}
Wallclock times and CPU usage values (in hh:mm:ss format) of the phrase table
construction for various experiments of "Fr-En" setup.}
\end{table}

...

\Tref{fr-en-vm-and-disk-usage-peaks} illustrates both memory and disk space demands of baseline
and \eppex{} experiments.

% Fr-En: virtual memory and disk usage peaks
\begin{table}[ht]
\centering
\begin{tabular}{ | c | r l | c | }
\hline
 & \multicolumn{2}{|c|}{VM peak} & \\
experiment & size & step & disk usage peak \\
\hline
\hline
opt-base       & 200.1~GB & extraction & 160.5~GB \\
opt-c8-base    & 200.1~GB & extraction & 160.5~GB \\
eppex zero     & 203.0~GB &      eppex &  48.0~GB \\
\hline
eppex def.     &  89.9~GB &      eppex &  16.0~GB \\
eppex zero-n   &  48.7~GB &      eppex &   8.0~GB \\
eppex 1:n+1    &  46.4~GB &      eppex &   4.0~GB \\
eppex 2:n+2    &  45.6~GB &      eppex &   2.0~GB \\
\hline
\end{tabular}
\caption{\label{fr-en-vm-and-disk-usage-peaks}
Virtual memory peaks of the phrase table construction for various experiments of "Fr-En" setup.}
\end{table}

...


\section{Comparison to 2011 version}
\label{sec:cu-bojar-results}
% Comparison of current eppex and phrase-extract performance vs. mid-2011 state

\citet{przywara:eppex} presented an evaluation of an early version of \eppex{}
that was able to only extract phrases. It performed a faster phrase
table creation than the legacy \emph{phrase-extract} in such situations,
where lossy counting resulted in significant filtration of extracted phrase
pairs and consequent sub-steps (sorting, scoring and consolidation of both
phrase table halves) had to process reduced amount of data;
eventually the whole process of phrase table creation finished faster.

As we had all the parallel data used in their experiments available and
we perform our runtime benchmarking in almost identical manner,
we decided to carry out comparison between early and current version of
both \eppex{} as well as \emph{phrase-extract} toolkit.

\subsection{Implementation differences}

The current version of \eppex{} differs from the early version mainly in the
capability of performing not only phrase pairs extraction, but a complete
phrase table construction (that is including phrase pairs scoring).
The early version, however, could be used to extract orientation info
the same way \emph{extract} can, while this functionality has been dropped
from the current version as not related to the core task of phrase table creation.
Following the upgrade of \emph{phrase-extract} suite, an option to read from
gzipped input files and dump gzipped output files has been added to \eppex{}.
Internally, there have been a multitude of performance tweaks,
as performance optimization is the most important aim in \eppex{} development,
but a full listing of implementation changes would be purposeless here.

On the contrary, \emph{phrase-extract} suite has been updated mainly in order to
provide a richer functionality, but three changes since the mid-2011 version
had sound impact on its runtime performance: the optimization of target phrases
scanning in \emph{scorer} implementation,\footnote{Commit 677378774aca30c8f0d4ca57267f7ac5ef7d7cb6.}
adding an option to gzip the output directly within the main three binaries
(\emph{extract}, \emph{scorer} and \emph{consolidate})
and parallelization of both phrase extraction and phrase scoring
steps.\footnote{See \texttt{extract-parallel.perl} and \texttt{score-parallel.perl}
in \texttt{<moses>/scripts/generic/}.}

\subsection{Parameters of experiments}

\Tref{cu-bojar-scenarios} presents all the experiments and their settings.
Again, we have included one more optimized baseline with 7 cores, because
the achieved speeding up is considerable.
The \emph{default baseline} is invoked with the same parameters as \emph{2011
baseline}, except for an option that turns on gzipping of temporary data and
phrase table and is now implicitly activated from within the training script.
Because of this, we also turn \verb|--GZOutput| on for all \eppex{} runs.
We experimented with different pruning parameters: \emph{eppex-1-in} (milder
pruning) and \emph{eppex-1-out} (harsher pruning) are using the same pruning
parameters as 2011 experiments and we also perform \emph{eppex-zero} experiment
(an \eppex{} run with no pruning).

% Detailed info of cu-bojar experiments
\begin{table}[ht]
\centering
\begin{tabular}{ r p{10cm} }
name & description and parameters \\
\hline
\hline
def-base        & Standard Moses pipeline with no special parameters \\
multi-base      & Standard Moses pipeline with \verb|--cores 4| \\
comp-base       & Standard Moses pipeline with \verb|--sort-compress gzip|
  and \verb|--cores 4| \\
opt-base        & Standard Moses pipeline with \verb|--sort-buffer 12G|,
  \verb|--sort-compress gzip| and \verb|--cores 4| \\
opt-c7-base     & Standard Moses pipeline with \verb|--sort-buffer 12G|,
  \verb|--sort-compress gzip| and \verb|--cores 7| \\
eppex zero      & \eppex{} set to no pruning and \verb|--GZOutput| option \\
eppex 1-in      & \eppex{} with pruning thresholds set to keep in
  all phrase pairs of length 1--3 and prune longer phrase pairs
  with max. positive threshold of 8 and \verb|--GZOutput| option \\
eppex 1-out     & \eppex{} with pruning thresholds set to remove
  all single-occurring phrase pairs and prune the rest with
  max. positive threshold of 8 and \verb|--GZOutput| option \\
\hline
\hline
\end{tabular}
\caption{\label{cu-bojar-scenarios}List of various experiments and their
settings for "cu-bojar" setup. The parameters of "eppex 1-in" and "eppex 1-out"
conform to the corresponding 2011 experiments, "def-base" corresponds to 2011
"baseline" experiment except for (now activated implicitly) gzipping of
temporary data.}
\end{table}

\subsection{Memory and time requirements}

% NOTE:
% Steps and substeps of phrase table construction via train-model.perl back in 2011:
% (1) Phrase extraction
%  a) extract
%  b) gzip f2e
%  c) gzip e2f
% (2) Phrase scoring
%  d) sort f2e (now done in (1) and so reported as part of phrase extraction in table below)
%  e) score e2f
%  f) sort e2f (now done in (1) and so reported as part of phrase extraction in table below)
%  g) score e2f
%  h) sort inv
%  i) cons
%  j) gzip pt

\Tref{cu-bojar-time-benchmarks} compares wallclock times and CPU usage of all the
experiments and in case of baseline and 2011 experiments also separately for phrase
extraction and phrase scoring steps.

The comparison between old and current baseline reveals that the optimization of
\emph{scorer} mentioned above resulted in a speed up by more than a factor of two.
On the other hand the default phrase extraction became slightly more time demanding,
but this may be explained by the fact that parallelization incurred some overhead
that does not pay back when running only with a single core, but significantly cuts
down the running time when using multiple cores: in our setup using 7~cores (along with
more memory for sorting) lowered time of phrase extraction to 40\% and of phrase scoring
to almost 50\% of respective default measures.

The comparison between 2011 and current version of \eppex{} also display a significant
speed up: with harsh pruning the phrase table construction is done in 30~minutes (took
almost two hours in 2011) and without any pruning the full phrase table was built in
less than 90~minutes, thus \eppex{} proves to be a viable alternative to
\emph{phrase-extract} also in situations, when pruning is not an option (being twice as
fast than our optimized baseline).

\begin{table}[ht]
\centering
\begin{tabular}{ | c | c c | c c | c c | }
\hline
 & \multicolumn{2}{|c|}{total time} & \multicolumn{2}{|c|}{phrase extraction} & \multicolumn{2}{|c|}{phrase scoring} \\
experiment & wall & CPU & wall & CPU & wall & CPU \\
\hline
\hline
baseline*     & 08:49:44 & 07:03:48 & 02:05:56 & 01:00:09 & 06:43:48 & 06:03:39 \\
def-base      & 05:36:33 & 04:45:12 & 02:29:04 & 01:31:17 & 03:07:29 & 03:13:55 \\
multi-base    & 04:15:51 & 05:39:02 & 02:11:56 & 01:32:03 & 02:03:54 & 04:06:59 \\
comp-base     & 03:20:29 & 07:03:40 & 01:03:20 & 02:30:16 & 02:17:09 & 04:33:23 \\
opt-base      & 03:14:51 & 05:57:08 & 01:08:29 & 01:39:06 & 02:06:21 & 04:18:01 \\
opt-c7-base   & 02:47:19 & 05:57:51 & 01:03:50 & 01:38:30 & 01:43:28 & 04:19:21 \\
eppex zero    & 01:23:28 & 01:23:26 &       -- &       -- &       -- &       -- \\
\hline
eppex 1-in*   & 04:03:04 & 03:42:29 & 01:48:14 & 01:28:53 & 02:14:50 & 02:13:36 \\
eppex 1-in    & 00:45:14 & 00:45:13 &       -- &       -- &       -- &       -- \\
\hline
eppex 1-out*  & 01:50:03 & 01:36:19 & 01:35:38 & 01:21:57 & 00:14:25 & 00:14:22 \\
eppex 1-out   & 00:30:29 & 00:30:28 &       -- &       -- &       -- &       -- \\
\hline
\end{tabular}
\caption{\label{cu-bojar-time-benchmarks}
Wallclock times and CPU usage values (in hh:mm:ss format) of the phrase table
construction for various experiments of "cu-bojar" setup.
Asterisk denotes the particular measures from 2011 experiments.}
\end{table}

\Tref{cu-bojar-vm-and-du-peaks} presents memory and disk usage peaks of all the experiments
(except for disk usage peaks that are not available for 2011 experiments).

Memory demands of the baseline keeps to be low, the 0.1~GB difference between 2011 and now
stems from the fact that our benchmarking script includes the main process into the set of
measured processes.
An optimized baseline is the most memory demanding, but this is only because we assigned 12~GB
of RAM to the sorting processes and both phrase table halves are sorted simultaneously after
phrase extraction.

Memory demands of current version of \eppex{} are significantly lower than of the 2011 version:
in 2011, the harsh pruning setup required as much memory as the current version run without
any pruning at all.

The disk peak usage is significantly reduced in \emph{compressed} baseline, but rises almost to
default levels in optimized baselines, despite the \verb|--sort-compress| option has been applied
to them as well.
We do not poses sufficient knowledge of the inner workings of \emph{GNU sort}, therefore we do not
attempt to give an explanation of this peculiar behavior.

\begin{table}[ht]
\centering
\begin{tabular}{ | c | c | c | }
\hline
experiment & VM peak & disk usage peak \\
\hline
\hline
baseline*     &  1.1~GB &      -- \\
def-base      &  1.2~GB & 51.8~GB \\
multi-base    &  3.9~GB & 53.6~GB \\
comp-base     &  3.9~GB & 14.3~GB \\
opt-base      & 24.1~GB & 43.7~GB \\
opt-c7-base   & 24.1~GB & 43.7~GB \\
eppex zero    & 16.8~GB &  3.7~GB \\
\hline
eppex 1-in*   & 19.2~GB &      -- \\
eppex 1-in    & 13.4~GB &  1.3~GB \\
\hline
eppex 1-out*  & 16.7~GB &      -- \\
eppex 1-out   & 11.3~GB &  0.3~GB \\
\hline
\end{tabular}
\caption{\label{cu-bojar-vm-and-du-peaks}
Virtual memory and disk usage peaks of the phrase table construction for various experiments
of "cu-bojar" setup. Asterisk denotes the particular measures from 2011 experiments.}
\end{table}
