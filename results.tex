\chapter{Results}
\label{chap:results}

\section{Comparison to 2011 version}
% Comparison of current eppex and phrase-extract performance vs. mid-2011 state

The early version of \eppex{} was able to only extract phrases and therefore
perform a faster phrase table creation than the legacy \emph{phrase-extract}
only in situations, where lossy counting resulted in significant filtration of
extracted phrase pairs, so the consequent sub-steps (sorting, scoring and
consolidation of both phrase table halves) had to process reduced amount of
data and the whole process eventually finished faster.
\citet{przywara:eppex} presented an evaluation of that early version of
\eppex{} (we will refer to it as "2011 experiments" in further text).

As we had all the parallel data used in 2011 experiments available and
we perform our runtime benchmarking in almost identical manner,
we decided to perform comparison between early and current version of both
\eppex{} as well as \emph{phrase-extract} toolkit.

\subsection{Implementation differences}

The current version of \eppex{} differs from the early version mainly in the
capability of performing not only phrase pairs extraction, but a complete
phrase table construction (that is including phrase pairs scoring).
The early version, however, could be used to extract orientation info
the same way \emph{extract} can, while this functionality has been dropped
from the current version as not related to the core task of phrase table creation.
Internally, there have been a multitude of performance tweaks,
as performance optimization is the most important aim in \eppex{} development,
but a full listing of implementation changes would be purposeless here.

On the contrary, \emph{phrase-extract} suite has been updated mainly in order to
provide a richer functionality, but three changes since the mid-2011 version
had sound impact on its runtime performance: the optimization of target phrases
scanning in \emph{scorer} implementation,\footnote{Commit 677378774aca30c8f0d4ca57267f7ac5ef7d7cb6.}
adding an option to gzip the output directly within the main three binaries
(\emph{extract}, \emph{scorer} and \emph{consolidate})
and parallelization of both phrase extraction and phrase scoring
steps.\footnote{See \texttt{extract-parallel.perl} and \texttt{score-parallel.perl}
in \texttt{<moses>/scripts/generic/}.}

\subsection{Benchmarking results}

% Detailed info on configuration
\begin{table}[ht]
\centering
\begin{tabular}{ r p{10cm} }
Name & Description and parameters to \verb|train-model.perl| \\
\hline
\hline
baseline & Standard Moses pipeline with no special parameters \\
opt-baseline & Standard Moses pipeline with an optimized setup that is
available in Moses 1.0:
\verb|--sort-compress gzip|, \verb|--sort-buffer 12G| and \verb|--cores 7| \\
eppex-zero & \eppex{} set to no pruning and \verb|--GZOutput| option \\
eppex-1-in & \eppex{} with pruning thresholds set to keep in
all phrase pairs of length 1--3 and prune longer phrase pairs
with max. positive threshold of 8 and \verb|--GZOutput| option \\
eppex-1-out & \eppex{} with pruning thresholds set to remove
all single-occurring phrase pairs and prune the rest with
max. positive threshold of 8 and \verb|--GZOutput| option \\
\hline
\hline
\end{tabular}
\caption{\label{cu-bojar-scenarios}List of various experiments and their
settings for "cu-bojar" setup. The parameters of "baseline", "eppex-1-in"
and "eppex-1-out" are set in such a way that they conform to the
corresponding 2011 experiments.}
\end{table}

% NOTE:
% Steps and substeps of phrase table construction via train-model.perl back in 2011:
% (1) Phrase extraction
%  a) extract
%  b) gzip f2e
%  c) gzip e2f
% (2) Phrase scoring
%  d) sort f2e (now done in (1) and so reported as part of phrase extraction in table below)
%  e) score e2f
%  f) sort e2f (now done in (1) and so reported as part of phrase extraction in table below)
%  g) score e2f
%  h) sort inv
%  i) cons
%  j) gzip pt

\begin{table}[ht]
\centering
\begin{tabular}{ | c | c c | c c | c c | }
\hline
name & \multicolumn{2}{|c|}{total} & \multicolumn{2}{|c|}{phrase extraction} & \multicolumn{2}{|c|}{phrase scoring} \\
 & wall & CPU & wall & CPU & wall & CPU \\
\hline
\hline
baseline*    & 08:49:44 & 07:03:48 & 02:05:56 & 01:00:09 & 06:43:48 & 06:03:39 \\
baseline     & 05:41:19 & 04:48:27 & 02:31:53 & 01:33:39 & 03:09:25 & 03:14:48 \\
opt-baseline & 03:26:04 & 06:03:30 & 01:08:36 & 01:38:55 & 02:17:27 & 04:24:35 \\
eppex-zero   & 01:23:45 & 01:23:37 & -- & -- & -- & -- \\
\hline
eppex-1-in*  & 04:03:04 & 03:42:29 & 01:48:14 & 01:28:53 & 02:14:50 & 02:13:36 \\
eppex-1-in   & 00:45:27 & 00:45:24 & -- & -- & -- & -- \\
\hline
eppex-1-out* & 01:50:03 & 01:36:19 & 01:35:38 & 01:21:57 & 00:14:25 & 00:14:22 \\
eppex-1-out  & 00:30:24 & 00:30:21 & -- & -- & -- & -- \\
\hline
\end{tabular}
\caption{\label{cu-bojar-time-benchmarks}Wall clock and CPU time values (in hh:mm:ss format)
of the phrase table construction for various settings of "cu-bojar" setup.}
\end{table}

\begin{table}[ht]
\centering
\begin{tabular}{ | c | c | }
\hline
name & VM peak \\
\hline
\hline
baseline*    & 1.1~GB \\
baseline     & 2.4~GB \\ % TODO: forked train-model.perl takes up more then 1GB...
opt-baseline & 24.1~GB \\
eppex-zero   & 16.8~GB \\
\hline
eppex-1-in*  & 19.2~GB \\
eppex-1-in   & 13.4~GB \\
\hline
eppex-1-out* & 16.7~GB \\
eppex-1-out  & 11.2~GB \\
\hline
\end{tabular}
\caption{\label{cu-bojar-vm-peak-benchmarks}Virtual memory peak values
of the phrase table construction for various settings of "cu-bojar" setup.}
\end{table}
