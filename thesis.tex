\documentclass[12pt,a4paper]{report}
%% Verze pro jednostranný tisk:
% Okraje: levý 40mm, pravý 25mm, horní a dolní 25mm
% (ale pozor, LaTeX si sám přidává 1in)
\setlength\textwidth{145mm}
\setlength\textheight{247mm}
\setlength\oddsidemargin{15mm}
\setlength\evensidemargin{15mm}
\setlength\topmargin{0mm}
\setlength\headsep{0mm}
\setlength\headheight{0mm}
% \openright zařídí, aby následující text začínal na pravé straně knihy
\let\openright=\clearpage

% http://en.wikibooks.org/wiki/LaTeX/Paragraph_Formatting
% Use additional margin space instead of indented first line to mark new paragraph.
\usepackage{parskip}

%% Pokud tiskneme oboustranně:
% \documentclass[12pt,a4paper,twoside,openright]{report}
% \setlength\textwidth{145mm}
% \setlength\textheight{247mm}
% \setlength\oddsidemargin{15mm}
% \setlength\evensidemargin{0mm}
% \setlength\topmargin{0mm}
% \setlength\headsep{0mm}
% \setlength\headheight{0mm}
% \let\openright=\cleardoublepage

% Why use T1 for fontenc package: http://tex.stackexchange.com/questions/664/why-should-i-use-usepackaget1fontenc
\usepackage[T1]{fontenc}
% Nice intro to LaTeX fonts: http://www-h.eng.cam.ac.uk/help/tpl/textprocessing/fonts.html
% Better list of font families: http://www.macfreek.nl/memory/Fonts_in_LaTeX#Specifying_Fonts_to_Use
\usepackage{lmodern} % to set monospace to LM Typewriter
\usepackage{mathpazo} % to set roman to Palatino

%% Použité kódování znaků: obvykle latin2, cp1250 nebo utf8:
\usepackage[utf8]{inputenc}

%% Balíčky doporučené skrz http://repo.or.cz/w/csplainnat.git
\usepackage{natbib}             % sazba pouzite literatury
\usepackage{url}                % sazba URL

%% Ostatní balíčky
\usepackage{graphicx}
\usepackage{amsthm}
\usepackage{amssymb}

% Redefines the underscore symbol so that you don't have to escape it in text mode:
% http://stackoverflow.com/a/1346534
\usepackage{underscore}

%% Balíček hyperref, kterým jdou vyrábět klikací odkazy v PDF,
%% ale hlavně ho používáme k uložení metadat do PDF (včetně obsahu).
%% POZOR, nezapomeňte vyplnit jméno práce a autora.
\usepackage[ps2pdf,unicode]{hyperref}   % Musí být za všemi ostatními balíčky
\hypersetup{pdftitle=Streamed Phrase Table Extraction}
\hypersetup{pdfauthor=Česlav Przywara}
\hypersetup{colorlinks=true} % Better option for PDF file.
%\hypersetup{colorlinks=false} % Better option for print.

%%% Drobné úpravy stylu

% Tato makra přesvědčují mírně ošklivým trikem LaTeX, aby hlavičky kapitol
% sázel příčetněji a nevynechával nad nimi spoustu místa. Směle ignorujte.
\makeatletter
\def\@makechapterhead#1{
  {\parindent \z@ \raggedright \normalfont
   \Huge\bfseries \thechapter. #1
   \par\nobreak
   \vskip 20\p@
}}
\def\@makeschapterhead#1{
  {\parindent \z@ \raggedright \normalfont
   \Huge\bfseries #1
   \par\nobreak
   \vskip 20\p@
}}
\makeatother

% Toto makro definuje kapitolu, která není očíslovaná, ale je uvedena v obsahu.
\def\chapwithtoc#1{
\chapter*{#1}
\addcontentsline{toc}{chapter}{#1}
}

\newtheorem*{definition}{Definition}

% Ondřej's definitions
\def\url#1{{\tt{}#1}}
\def\footurl#1{{\footnote{\tt{}#1}}}
\def\Cref#1{Chapter~\ref{#1}}
\def\Fref#1{Figure~\ref{#1}}
\def\Sref#1{Section~\ref{#1}}
\def\Tref#1{Table~\ref{#1}}

% My definitions
\def\Aref#1{Appendix~\ref{#1}}
\def\Eppex{{\emph{Eppex}}}
\def\eppex{{\emph{eppex}}}


%%%%%%%%%%%%%%%%%%%%%%%%%%%%%%%%%%%%%%%%%%%%%%%%%%%%%%%%%%%%%%%%%%%%%%%%%%%%%%
%%%%%%%%%%%%%%%%%%%%%%%%%%%%%%%%%% DOCUMENT %%%%%%%%%%%%%%%%%%%%%%%%%%%%%%%%%%
%%%%%%%%%%%%%%%%%%%%%%%%%%%%%%%%%%%%%%%%%%%%%%%%%%%%%%%%%%%%%%%%%%%%%%%%%%%%%%

\begin{document}

% Trochu volnější nastavení dělení slov, než je default.
\lefthyphenmin=2
\righthyphenmin=2

%%% Titulní strana práce

\pagestyle{empty}
\begin{center}

\large

Charles University in Prague

\medskip

Faculty of Mathematics and Physics

\vfill

{\bf\Large MASTER THESIS}

\vfill

\centerline{\mbox{\includegraphics[width=60mm]{logo.eps}}}

\vfill
\vspace{5mm}

{\LARGE Česlav Przywara}

\vspace{15mm}

% Název práce přesně podle zadání
{\LARGE\bfseries Streamed Phrase Table Extraction}

\vfill

% Název katedry nebo ústavu, kde byla práce oficiálně zadána (dle Organizační struktury MFF UK)
Institute of Formal and Applied Linguistics

\vfill

\begin{tabular}{rl}

% TODO: Tesne pred tiskem overit, zda Bojar nema nejaky novy titul :)
Supervisor of the master thesis: & RNDr. Ondřej Bojar Ph.D. \\
\noalign{\vspace{2mm}}
Study programme: & Computer Science \\
\noalign{\vspace{2mm}}
Specialization: & Mathematical Linguistics \\
\end{tabular}

\vfill

% Zde doplňte rok
Prague 2013

\end{center}

\newpage

%%% Následuje vevázaný list -- kopie podepsaného "Zadání diplomové práce".
%%% Toto zadání NENÍ součástí elektronické verze práce, nescanovat.

%%% Na tomto místě mohou být napsána případná poděkování (vedoucímu práce,
%%% konzultantovi, tomu, kdo zapůjčil software, literaturu apod.)

\openright

\noindent
%TODO: Thank someone :)

\newpage

%%% Strana s čestným prohlášením k diplomové práci

\vglue 0pt plus 1fill

\noindent
I declare that I carried out this master thesis independently, and only with the cited
sources, literature and other professional sources.

\medskip\noindent
I understand that my work relates to the rights and obligations under the Act No.
121/2000 Coll., the Copyright Act, as amended, in particular the fact that the Charles
University in Prague has the right to conclude a license agreement on the use of this
work as a school work pursuant to Section 60 paragraph 1 of the Copyright Act.

\vspace{10mm}

\hbox{\hbox to 0.5\hsize{%
In .......... date ...............
\hss}\hbox to 0.5\hsize{%
Signature
\hss}}

\vspace{20mm}
\newpage

%%% Povinná informační strana diplomové práce

\vbox to 0.5\vsize{
\setlength\parindent{0mm}
\setlength\parskip{5mm}

Název práce:
Přibližná extrakce frázové tabulky z velkého paralelního korpusu
% přesně dle zadání

Autor:
Česlav Przywara

Katedra:  % Případně Ústav:
Ústav formální a aplikované lingvistiky
% dle Organizační struktury MFF UK

Vedoucí diplomové práce:
RNDr. Ondřej Bojar Ph.D., Ústav formální a aplikované lingvistiky
% dle Organizační struktury MFF UK, případně plný název pracoviště mimo MFF UK

Abstrakt:
% abstrakt v rozsahu 80-200 slov; nejedná se však o opis zadání diplomové práce - TODO

Klíčová slova:
% 3 až 5 klíčových slov - TODO

\vss}\nobreak\vbox to 0.49\vsize{
\setlength\parindent{0mm}
\setlength\parskip{5mm}

Title:
Streamed Phrase Table Extraction
% přesný překlad názvu práce v angličtině

Author:
Česlav Przywara

Department:
Institute of Formal and Applied Linguistics
% dle Organizační struktury MFF UK v angličtině

Supervisor:
RNDr. Ondřej Bojar Ph.D., Institute of Formal and Applied Linguistics
% dle Organizační struktury MFF UK, případně plný název pracoviště
% mimo MFF UK v angličtině

Abstract:
% abstrakt v rozsahu 80-200 slov v angličtině; nejedná se však o překlad
% zadání diplomové práce - TODO

Keywords:
% 3 až 5 klíčových slov v angličtině - TODO

\vss}

\newpage

%%% Strana s automaticky generovaným obsahem diplomové práce. U matematických
%%% prací je přípustné, aby seznam tabulek a zkratek, existují-li, byl umístěn
%%% na začátku práce, místo na jejím konci.

\openright
\pagestyle{plain}
\setcounter{page}{1}
\tableofcontents

%%% Jednotlivé kapitoly práce jsou pro přehlednost uloženy v samostatných souborech

% Introduction
\chapter{Introduction}
\label{chap:introduction}

\setlength{\epigraphwidth}{1.0\textwidth}
\epigraph{This page is in Czech. Would you like to translate it?}{--- Google}

% A catchy intro: too informal for thesis?
It would make for a nice brain-teasing question, of which job interviewers
at Google are famous\footnote{For the actual tricky questions and puzzles
asked by Google interviewers I recommend reading the book "Are You Smart
Enough to Work at Google?" by William Poundstone.}, to ask:
"How many bilingual elves would Google had to employ in order to translate
all the words, sentences and web pages that are send to Google Translate
service each day?"\footurl{http://translate.google.com}
Obviously, it would be a hypothetical one. No bilingual elves are involved
in translation of the bunch of texts sent to Google Translate each day.
And no human translators either: Google Translate is fully implemented using
computer software only.

% What is MT?
The approach Google Translate uses to deliver translations is described as
\emph{machine translation} (MT) and more formally can be defined as a design
or development of computer software capable of fully automated translation
of text or speech from one natural language to another.
The words "fully automated" are important here as machine translation should
not be confused with approaches that only aims to design computer programs
helping human translators to work more efficiently: these are typically
referred to as computer-aided translation (CAT) systems.

% A brief look into history.
The first attempts to use computers for human language translation date
back to the early ages of computing. \citet{weaver:memorandum} in his famous
memorandum suggested:
\begin{quote}
(...) the possibility of contributing at least
something to the solution of the world-wide translation problem through the use
of electronic computers of great capacity, flexibility, and speed.
\end{quote}
The outset of the MT field has been marked with immense optimism, leading
to prognosis that machine translation would be a solved problem within
a decade. % TODO: Cite?
This quickly proved to be unrealistic assumption and even nowadays
there is still no system that provides the holy grail of MT:
\emph{fully automatic high quality translation of unrestricted text}.

% Introduce SMT. TODO: Slightly inspired by Preface of Koehn's book.
However, the last 25 years have been very prolific for the field.
A new paradigm arose, not only in MT but in natural language processing
field in general, that employs automatic discovery of principles that
rules human languages (and translation between them) by collecting
statistics over the data rather than explicit definition of such rules
by human experts.
In \emph{statistical machine translation} (SMT) these statistics are
collected by pairing the input and output side of the translation process.
Co-occurrences of \emph{atomic units of translation} are used to
evaluate properties of statistical model that is afterwards used to
search for the most probable translation given the input text.
The atomic units of translation are typically words or phrases and
the respective systems are referred to as \emph{word-based} or
\emph{phrase-based}.\footnote{In SMT context the term phrase is almost
exclusively used to refer to any short sequence of words and wears no
implicit linguistic notion.}

% Introduce phrase table.
Phrase-based systems internally utilize a table consisting of pairs of
phrases, one being the phrase from the source language and the other
being the phrase from the target language, and various scores assigned to
these pairs by the statistical model.
This table acts as a dictionary, listing all possible translations of phrases
in one language into phrases in another language along with indication of
quality of such translation expressed by the scores, and is usually referred
to as \emph{phrase translation table} or just \emph{phrase table}.
Any reasonable metric can be used as a phrase table score, but a de facto
standard is to use \emph{maximum likelihood probability} of the target language
phrase given the source language phrase and vice versa.

% Sketch the problem (partially copy-pasted from eppex paper).
To estimate maximum likelihood probabilities, frequency counts of source
phrases, target phrases and all their co-occurrences must be collected from
the entire parallel corpus.
For substantial coverage of source and target languages, such corpus is often
very big and in consequence all phrase pairs and their counts cannot fit in
the physical memory of the computer.
To overcome this limitation, phrase table construction methods often simply
dump observed phrases to local disk and sort and count them on disk.
This approach allows to construct phrase tables of size limited only by
the capacity of the disk, with an obvious drawback that much more time is
needed to build the table.

% Introduce existing tools for phrase table filtration.
In recent years it has been demonstrated that the load of phrase pairs can
be much reduced without loss of translation quality:
\citet{johnson:sigfilter} demonstrated filtering method based on significance
testing of phrase pair cooccurrences in the parallel corpus that allowed for
substantial savings (up to 90\%) and caused no reduction in BLEU score
\citep{papineni:bleu}.
However, the phrase table construction time is unaffected by this method,
as it requires frequency counts of phrases and phrase pairs and therefore acts
as a post-filter applied to already built phrase table.

% TODO: Mention phrase table compacting?

\section{Aim of this work}

% Introduce our solution.
In this work, we examine capability of an algorithm that delivers approximate
frequency counts over stream of input items \citep{manku:lossycounting} to
work as on-the-fly filter applied to the phrase pairs extraction,
essentially speeding up the whole process and eliminating the need for
any post-filtering of created phrase table.
This approach has already been demonstrated as applicable by \citet{przywara:eppex}
and this thesis is a direct follow-up of that effort.

% What's our goal that should be confronted against in conclusions?
The ultimate goal of this work is to implement a software tool that performs
the filtrated phrase table construction using aforementioned algorithm.
A successful implementation should allow to process parallel corpora of
significant sizes (tens of millions of sentences) with memory demands manageable
by physical memories available on present computation servers (tens of GBs).
Beside the implementation and its detailed description, the crucial part of
this work also consists of a careful examination of the impact that various
settings of the algorithm imply on memory vs. time and translation quality
trade-offs when compared to current state-of-art methods of phrase table
creation and filtration.

As a state-of-art SMT system to set our baseline, we decided to choose
\emph{Moses}\footurl{http://www.statmt.org/moses/}, an open-source toolkit
with rich documentation and active community of researchers and developers.

\section{Thesis outline}

We start with more detailed introduction to the phrase-based SMT,
carefully describe the process of phrase translation table construction
and mention some of the existing filtration tools.

In Chapter 3 we introduce the algorithm that is the basis of our implementation
of on-the-fly filtration and show the properties of the output produced
by the algorithm that make it particularly applicable in the filtration process.

Chapter 4 is devoted to in-depth description of implementation details of our
phrase table extraction tool, \emph{epochal extractor} (or shortly \eppex{}).
Notably, various memory-management optimizations are mentioned.

To assess \eppex{} usability in real world applications, we carried out a set
of carefully crafted experiments aiming at comparison of resources usage as well as
the ultimate translation quality of \eppex{} and some of the methods mentioned
in Chapter 2.
Detailed design of experiments is subject of Chapter 5,
while the results are discussed in Chapter 6.

In last chapter we comment on our results and provide a conclusion of
what have been done and what can be done in the future work on this topic.

In Appendix A you may find \eppex{} installation instructions.

In Appendix B all the program options are explained with examples of their usage.


% A minor introduction to phrase based SMT and phrase table concept
% In this chapter:
% - introduction of phrase based SMT
% - description of phrase table
% - description of phrase table creation in Moses
% -- with alternatives: memscore
% - phrase table post filtering
% -- sigfilter
% - alternatives to precomputed phrase table
% -- dynamic suffix arrays
% - mention phrase table compression tools

\chapter{Phrase tables in phrase-based Statistical Machine Translation}
\label{chap:phrase-based}

% Briefly introduce the concept of phrase-based SMT
% (SMT book, Chapter 5)

Phrase-based models are nowadays considered the most eminent approach in
the field of Statistical Machine Translation and the best performing
SMT systems are based on phrase-based models \citep{koehn:smt},
Google Translate\footurl{http://translate.google.com} being a prominent example.
Their mathematical motivation is very simple, yet they prove to be powerful enough
to compete with linguistically motivated rule-based models.
In addition, they conceptually benefit from steadily increasing amount of both
monolingual and multilingual corpora available as a positive side-effect of rapidly
growing usage of information technologies in our daily lives.

\section{Phrase-based Statistical Machine Translation}

Before phrase-based model were introduced, a yet simpler word-based models were in
the center of interest of MT researchers. \emph{Word-based models} use a lexicon
of words with their translations to produce translation of the full text by proceeding
with one word at a time.
This simple approach works well, but has obvious drawbacks,
for example the inability to catch word context.
Also it does not account for situations, where foreign word has to be translated
with two or more words or vice versa.

\emph{Phrase-based models} originates from the concept of word-based models,
but instead of using only single word as translation unit they
allow for sequence of words -- \emph{a phrase} -- to be translated at once.
This seemingly trivial extension makes for a non-trivial improvement
with following imminent benefits:
\begin{enumerate}
  \item By using phrases rather than words the model is now implicitly able to deal with
    \emph{one-word-translates-to-many} situations.
  \item A phrase-based model is able to handle local reordering of words for example switching
    from \emph{noun follows adjective} to \emph{adjective follows noun} order etc.
  \item Word context on source side plays now more important role, because it might be kept
    within the phrase pair and lead to implicit disambiguation during translation.
  \item With a lot of data and storage space available the model can be trained to cover
    (and later easily translate) the whole sentences.
\end{enumerate}


\subsection{Mathematical Definition}
% TODO.


\section{Phrase translation table}
% What's the phrase table?

\emph{Phrase translation table} in SMT systems captures the mapping
between phrases in source language and phrases in target language
along with scores that should reflect the quality of translating a particular
source language phrase as a particular target language phrase.
The concept of phrase table is not related to any linguistic notion,
neither the particular phrase pair nor the scores have to be meaningful
in terms of syntax or semantics. The current phrase-based systems construct
phrase tables using \emph{maximum likelihood} estimates calculated from (almost)
plain text input data (hence statistical). The utility of translating phrase
$s$ into phrase $t$ is then represented by maximum likelihood probability of
the phrase $t$ given the phrase $s$.

An example of a simple phrase table is presented by \Tref{phrase-table-example}.
The table contains various options of how to translate Czech phrase
"pes" (dog) into English.

% TODO: Include real phrase table example.
% TODO: Include a phrase, not a single word only.
\begin{table}[h]
\centering
\begin{tabular}{ l l l}
Czech (s) & English (t) & Probability p(t|s) \\
\hline
\hline
pes & dog & 0.8 \\
pes & cat & 0.1 \\
pes & wolf & 0.07 \\
pes & hamster & 0.03 \\
\hline
\hline
\end{tabular}
\caption{\label{phrase-table-example}Phrase translation table extract.}
\end{table}

We may note that the example phrase table contains some erroneous items,
as dog is unlikely to become cat just by switching to a different language.
The fact is that standard methods of phrase table construction may produce such
unlikely phrase translations, one reason being that their input data usually
contains a lot of noise.
We will discuss this problem further in the following sections.

% TODO: Ctrl+C Ctrl+V from the eppex paper.

Phrase tables in Statistical Machine Translation (SMT) systems generally take
the form of a list of pairs of phrases $s$ and $t$, $s$ being the phrase from
the source language and $t$ being the phrase from the target language, along
with scores that should reflect the goodness of translating $s$ as $t$.
The standard approach to obtain such scores is to use \emph{maximum likelihood
probability} of the phrase $t$ given the phrase $s$ and vice versa.
The probabilities $p(s|t)$ and $p(t|s)$ are often referred to as
\emph{forward} and \emph{reverse} \emph{translation probabilities}.


\subsection{Phrase table creation}

% TODO: Ctrl+C Ctrl+V from the eppex paper.

To estimate $p(s|t)$ and $p(t|s)$, frequency counts $C(t,s)$, $C(s)$ and
$C(t)$ are usually collected from the entire training corpus.
For substantial coverage of source and target languages, such corpora are
often very big so all phrase pairs and their counts cannot fit in the
physical memory of the computer.
To overcome this limitation, phrase table construction methods often simply
dump observed phrases to local disk and sort and count them on disk.
This approach allows to construct phrase tables of size limited only by the
capacity of the disk.
The obvious drawback of this solution is that much more time is needed
to build the table.

% What's the state-of-art implementation of phrase table extraction in Moses?
% Introduce train-model.perl and steps of training pipeline.

Moses comes with a training script that incorporates all the steps involved in
creation of ready-to-go translation system from the parallel corpus,
including phrase table creation.
In the \emph{train-model.perl} script this whole training process is covered by
9 subsequent steps\footurl{http://www.statmt.org/moses/?n=FactoredTraining.HomePage},
including word alignment, lexical table construction, phrase table construction and more.

The modular step-by-step design of the training process makes it very open to
alternate implementations of particular steps.
The training script interface explicitly supports use of third party tools by
allowing only subrange of the nine steps to be executed.
The only (obvious) requirement is that such tools have to be capable of reading and
writing data in format that is compatible with the preceding and subsequent steps
of the pipeline.

In this work we are exclusively concerned with the fifth and sixth step of
the training pipeline: the phrase pairs extraction and phrase pairs scoring.
Conceptually, these two steps can be regarded as a single step only, as their ultimate
goal is to construct phrase table given parallel corpus and word alignment.
The reason why the phrase table construction is split in these two steps is that
for large input data the table simply does not fit into computer memory.
To overcome this limitation, the core implementation uses disk space as temporary storage
for extracted phrases and keeps only the minimum required data in random access memory.
In the remaining part of this section we describe this implementation in more detail.

...

\subsection{Phrase table pruning}

% Introduce Johnson's significance filtering.

\subsection{Phrase table compression}

% Introduce phrase table compacting tool by Marcin.


% Lossy Counting algorithm description
\chapter{Lossy Counting algorithm}
\label{chap:lossy-counting}

The Lossy Counting (LC) algorithm \citep{manku:lossycounting} is a \emph{deterministic algorithm}
that computes approximate frequency counts over virtually infinite stream of input items.
Although the counts are approximate, the error is guaranteed not to exceed a user-specified parameter.

Specifically, Lossy Counting algorithm expects to be given two parameters:
\emph{support} $s \in (0,1)$ and \emph{error} $\epsilon \in (0,1)$, such that $\epsilon \ll s$.
At any point of time (after being fed with $N$ items from the stream) the algorithm can be
asked to output the list of items with their approximate frequencies that satisfy the following guarantees:
\begin{itemize}
 \item All items whose true frequency exceeds $sN$ are output (\emph{no false negatives}).
 \item No item whose true frequency is less than $(s - \epsilon)N$ is output (\emph{few false positives}).
 \item Estimated frequencies are less than the true frequencies by at most $\epsilon N$.
 \item The space used by the algorithm is $O(\frac{1}{\epsilon}~log \epsilon N)$.
\end{itemize}

% TODO: Ctrl+C Ctrl+V form the eppex paper.

Conceptually, the Lossy Counting algorithm divides the incoming stream of items
into epochs of fixed size $w = \lceil \frac{1}{\epsilon} \rceil$
(thus the name \emph{epochal extraction}).
In order to deliver the frequency estimates, the algorithm maintains a data structure $D$
consisting of triples ($e$, $f$, $\Delta$), where $e$ is an element from the
stream, $f$ is its estimated frequency and $\Delta$ is the maximum possible
error in $f$. When a new item $e$ arrives, a lookup for $e$ in $D$ is
performed. If $e$ is already present, its frequency $f$ is incremented by one.
Otherwise a new triple ($e$, 1, $T-1$) is added to $D$, where $T$ denotes the
ID of current epoch (with IDs starting from 1).

At the end of each epoch (determined by $N \equiv 0~mod~w$), the algorithm
prunes off all items whose maximum true frequency is small. Formally, at the end
of the epoch $T$, all triples satisfying the condition $f + \Delta \leqslant T$ are
removed from $D$. When all elements in the stream have been processed, the
algorithm returns all triples ($e$, $f$, $\Delta$) where $f \geq
(s-\epsilon)N$.

The idea behind the algorithm is that frequent elements show up more than once
within each epoch so their frequencies are increased enough to survive the
filtering.

\section{...}

The positive side effect of  for the phrase table extraction task is
obvious: low frequent items will be discarded, while frequent ones will be kept, all this
leading to more compact, but quality phrase table.


% Eppex - implementation and usage
% In this chapter:
% - introduction of eppex
% -- design goals
% -- distinction from legacy phrase-extract tools
% - implementation details
% -- Murmur hash, boost pools, std c++11 hash tables, indexed storages, gzipped I/O

\chapter{Eppex}
\label{chap:eppex}

\setlength{\epigraphwidth}{1.0\textwidth}
\epigraph{Your program can always have a twice as much time to run, but not always have a~twice as much memory.}{--- Rudolf Kryl, Introduction to Programming course}

\Eppex{} is phrase pairs extraction and scoring tool capable of obtaining approximate
frequency counts of extracted phrase pairs by using Lossy Counting algorithm
(thus the name \eppex{}, an acronym for \emph{epochal phrase pairs extraction}).
It is designed to be an alternative to standard phrase extraction and scoring tools that
are shipped with Moses, implementing most of the functionality of steps 5 and 6 of
\emph{train-model.perl} script.
\Eppex{} input and output interface is fully compatible with those of the replaced tools
and \eppex{} in fact is intended to be invoked from within the Moses training script itself
by passing specific parameters.

\Eppex{} differs from its core counterparts in one important aspect: during its runtime
only physical memory is utilized, no temporary files are stored on disk as with \emph{extract}
and \emph{score} tools.
The goal is to make \eppex{} a faster alternative, aiming at environments with plenty of RAM.
Benchmarking of time/memory trade-offs was fundamental part of this work and the results are
thoroughly examined in \Cref{chap:results}.

\Eppex{} memory demands may be limited by setting more restrictive support and error thresholds
for Lossy Counting, but aggressive pruning may lead to loss of phrase table quality --
experimentally evaluated trade-offs are also discussed in \Cref{chap:results}.

This chapter is fully devoted to technical aspects of implementation and it expects the reader
to be familiar with the basics of programming and the C++ programming language, including
a basic knowledge of Standard Template Library (STL).
Instructions on how to install \eppex{} on Linux-based operating system are given in
\Aref{chap:installation}, usage instructions are given in \Aref{chap:usage}.

\section{Design goals and philosophy}

% What and why is a target platform?
\Eppex{} is implemented as a command-line program and, as indicated above, it is written in C++.
From the development perspective, \eppex{} adheres to the philosophy of Moses, which is mainly
reflected by the selection of development platform and third party libraries:
\begin{enumerate}
  \item The primary development platform is Linux --
  \eppex{} has been developed and tested on desktop version of Ubuntu 12.04 LTS with GCC 4.6.3 and
  server version of Ubuntu 10.04 LTS with GCC 4.4.3 installed.
  \item \Eppex{} relies on Boost C++ library\footurl{http://www.boost.org/} -- it is used to provide
  some essential functionality, so \eppex{} cannot be compiled without access to some of Boost
  headers and packages.\footnote{See \Aref{chap:installation} for details.}
\end{enumerate}
Nevertheless, a considerable effort has been taken to make the implementation as cross-platform
compatible as possible.

% 64-bit or 32-bit?
\Eppex{} is intended to be run on 64-bit machines, but this is rather a matter of fact than
a requirement: to process a parallel corpus of reasonable size, \eppex{} will in typical
setting require much more memory than 32-bit environments can provide\footnote{In 32-bit
environment the virtual address space holds no more than $2^{32}$ addresses, so at maximum
4,294,967,296 bytes (4~GB) of memory are utilizable. Workarounds exist, but they might be
considered unfeasible nowadays, as 64-bit architecture is well established.} (see
\Sref{sec:eppex-memory-demands} for some experimentally established figures).
Therefore, although not tested, \eppex{} should run as smoothly on 32-bit machine as it runs
on 64-bit, but the amount of input data that it will manage to process will have to be decent
or aggressive pruning will have to be undertaken.

\section{Effective C++}

% Why C++?
C++ is a programming language that offers a wide range of optimization techniques to
tune up both the execution speed and memory requirements of a program.
Both aspects are important to us, but memory usage optimization is our primary concern,
since the execution speed, when compared to legacy tools, is implicitly boosted by
the Lossy Counting algorithm and the fact that we keep all temporary data in the computer
memory instead of disk.
Besides, as pointed out in epigraph on the beginning of this chapter, memory, unlike time,
is a limited resource.
In this section we present several universal rules and techniques for effective C++
programming, that we followed and applied during implementation of \emph{eppex}.

\subsection{Data alignment and structure packing}

The issue behind \emph{data alignment} is best explained using example
-- let us define a dummy structure \verb|LossyCounterItem|:

\begin{verbatim}
struct LossyCounterItem {
  unsigned char maximum_error; // sizeof(unsigned char) == 1
  unsigned int frequency; // sizeof(unsigned int) == 4
  unsigned short item_id; // sizeof(unsigned short) == 2
};
\end{verbatim}

Given the definition above and with respect to the byte sizes of individual members,
one could expect that \verb|sizeof(LossyCountItem)| equals to 7, but this guess
will be wrong in most of the cases as the most likely result is 12.

This unexpected behavior stems from the fact that many machines require that
objects of certain type are aligned on architecture-dependent boundaries,
for example integers are often allocated on word\footnote{\emph{Word} in computer
architecture terminology denotes a fixed-sized group of bits that are handled
as a unit by the instruction  set and/or hardware of the processor.} boundaries
\citep[Chapter 5]{stroustrup:cplusplus}.

The easiest way how to minimize effects of data alignment is to order members
by their size with the largest members first.
This way our \verb|LossyCounterItem| structure can be shrunk to 8 bytes,
but still there will be one byte of blank padding at the end of the structure.
With some compilers this padding can be trimmed by setting on \emph{structure packing}
using \verb|#pragma| preprocessor directive, but doing this will slow data access
on most processors that read memory by words, because misaligned structures will span
multiple words and in consequence require more reads.

This being said, the best solution in terms of code portability and effectiveness
is to design data structures in such a way that all their members properly align:
this ensures their effective handling by processing unit without spurious memory padding.

\subsection{C-string vs. std::string}
\label{sec:strings}
% Why had we used C-string instead of std::string.

Unlike Java, C and C++ have no fundamental data type for strings, instead they have
to be stored as an array of \verb|chars|.
Such string handling is obviously too low-level for most modern applications,
which are mainly about string processing, so STL creators came to help and designed
several \emph{string} classes that mimic behavior of ordinary data types
\citep[Chapter 11]{josuttis:stl}:
STL strings may be copied, assigned and compared like any other fundamental type,
without the programmer being worried about the internal memory allocations and deallocations.
Because both approaches are very often used side by side, there exists well-established
terminology to distinguish between them:
the arrays of characters are called \emph{C-like-strings} (or shortly \emph{C-strings}) and
STL string classes are just \emph{strings} (sometimes labeled \emph{std::string} after the
most commonly used class).

From the memory management standpoint it turns out, that any std::string
implementation will be more memory demanding than plain C-string.
This is expectable: the STL string class has to provide more flexible interface
and does all the internal memory management by itself.
Depending on particular implementation, it might store the size of the string,
store the capacity of allocated memory, do the reference counting and more
\citep[Item 15]{meyers:effectivestl}.
The reference counting feature seems to be useful for our phrase-counting problem,
but it would imply to rely on proprietary implementation that is not enforced by
standard, which is a bad design choice for any open-source software.

Therefore, whenever memory footprint of strings storage is more important than
the convenience and flexibility of their handling, C-string is a better choice
than any STL string implementation.

\subsection{Objects unification}
\label{sec:objects-unification}

When processing large stream of items that have the property of being distinguishable
either as different or equal to each other, it is often efficient to keep only a single
copy of each \emph{unique} item in computer memory and use some type of lightweight
reference to this copy in all the program parts that has to access item data.
For example, when counting $n$-grams in a text corpus, instead of keeping $n$ strings
within each $n$-gram object, it might be more efficient to just keep $n$ references to
unique, separately stored copies of such strings.\footnote{An observant reader may realize
that the task of items counting itself implicitly requires objects unification. In fact,
the example was deliberately picked to demonstrate that there are two aspects to object
unification: it is inherent in case of some tasks, such as items counting, but also more
generally applicable as a technique to achieve some memory savings.}
Obviously, this technique of \emph{objects unification} is not applicable universally and,
in general, its efficiency depends on the following factors:
\begin{enumerate}
  \item The input redundancy of items -- the greater the redundancy, the greater
  the possible memory savings. If there are no redundant items, this technique is useless,
  because it is cheaper to just store the item without an additional reference to it.
  \item The processing redundancy of items -- the more the items have to be
  referenced within the program, the greater the possible memory savings.
  \item The ratio between average in-memory size of item and reference to it -- the greater
  the ratio, the more memory is saved by each reference.
  Clearly, this ratio has to be greater than one.
\end{enumerate}

Typically, a programmer cannot affect the first factor, but should carefully consider it.
For example, in case of words pulled from natural language corpus it is generally hard to
presume whether their unification in program memory will be efficient:
Zipf's law \citep[Chapter 1]{manning:stat-nlp} suggests that the efficiency of words unification
will largely depend on the overhead imposed by the referencing scheme and if the references are
too heavy, the memory spared thanks to the most frequent words can be easily depleted by
the massive amount of words occurring only once.

The second factor might be tricky to consider in programs with non-linear work flow,
for example when the amount of items within program scope depends on some external
factor (time of day etc.), but in case of streamline input processing this value is
most often linear with respect to the input size.

The third factor is the only one that is completely in hands of the programmer.
With C++ we have two basic options, how to reference an object:
\begin{enumerate}
  \item by using object's address within memory
  \item by using a numeric index that uniquely identifies an object within a large storage
  (either a random access container or array).
\end{enumerate}

We may note that both options are fundamentally the same:
the pointer is basically a numeric index into the container of a maximum size (entire memory)
and the numeric index is just a pointer into restrained part of memory (container).
The pointer is easier to dereference, but comes in single size only -- the size of
\emph{processor unit word}, which on 64-bit architecture is 8~bytes long.
On the other hand, to dereference the numeric index, a corresponding container must be
accessed to determine the address of index start.
However, a numeric index data type can be adjusted to the size of domain of unified items,
if it is known in advance or can be assessed by some upper bound.
Being able to reference 4~billions of unique objects is enough in many situations and
in such cases only 4~bytes long integer type will suffice, reducing referencing overhead
to half of that of the pointer.

As a final note, we shall point out that objects unification comes with a cost of
additional processing overhead: especially insertion of new object is more time-consuming,
as every new object have to be first checked for an existing copy.

\subsection{Simple segregated storage}

Almost every non-trivial computer program uses dynamic memory allocation.
C++ provides two interfaces to ask for a new memory during program runtime:
one can either use legacy C API function \emph{malloc} or pure C++ API operator \emph{new}.
Both APIs has to provide means to satisfy any type of memory request -- and for any size.
Hence, usually the size of allocated memory block (and any other bookkeeping information) is
kept in memory chunk right before the block itself \citep[Chapter 10]{meyers:effective-cpp}.
This way, for deallocation either via \emph{free} function or operator \emph{delete}
memory allocator only needs the pointer to memory block that should be deallocated and
can easy determine, how much memory should be actually freed.

The obvious overhead in memory consumption incurred by this approach is usually negligible,
but may become significant in situations, when a load of small-sized memory blocks is allocated.
To avoid the overhead in such situations, a frequently used approach is to allocate big pool of
fixed-size memory blocks once a time and serve these blocks sequentially as new memory requests
come in, allocating a new pool whenever all memory blocks in existing pools are exhausted.
This technique is generally called \emph{memory pooling}, but when details of
memory pool allocation and deallocation matters, usually more specific names are used to
distinguish between various implementations --  a \emph{simple segregated storage} from Boost
library is an example of one such specific
implementation.\footurl{http://www.boost.org/doc/libs/1_53_0/libs/pool/doc/html/boost_pool/pool/pooling.html}

% Unions - in current version of Eppex they don't play significant role.

\subsection{Unordered sets}

The recent C++ standard,\footnote{ISO/IEC 14882:2011} marked as C++11, standardizes
a new type of containers, that were already part of almost every implementation of STL,
but due non-existing standardization they differ slightly in their interfaces across
various implementations.
The \emph{unordered associative containers} are well known to almost every programmer,
although usually they are referred to under a different name: \emph{hash tables}.

The C++11 standard defines four types of hash table containers that correspond to existing
binary-search-tree-based containers (with an \emph{unordered_} prefix):
\emph{unordered_set}, \emph{unordered_multiset}, \emph{unordered_map} and \emph{unordered_multimap}.
The key differences between hash tables and binary-search trees are:
\begin{itemize}
  \item Binary-search trees require less-than comparison operator to be defined for stored items,
  whereas hash tables require equal-to comparison operator and also a hash function to be
  defined for them.
  \item When iterating over binary-search tree the items are returned in order provided by
  less-than comparison operator, whereas iteration over hash table returns items in arbitrary
  order.
  \item Lookup, insertion and deletion operations with hash tables have amortized (constant)
  average cost per operation (one call to hash function plus one or more equality comparisons),
  whereas with binary-search tree these operations have cost dependent on the number of items
  in the tree ($N$): each operation requires $log_2(N)$ comparisons.
\end{itemize}

In the real life applications hash tables often perform much better than binary-search
trees, although their worst case complexity is linear with respect to number of items
(standard requires that hash tables are implemented using
\emph{closed addressing}\footurl{http://en.wikipedia.org/wiki/Hash_tables\#Separate_chaining}).
% TODO: Cite wikipedia or find a "better" source?
The effectiveness of hash tables stands and falls with a hash function:
a good hash function has to be fast to compute and display the quality of distributing
the data as evenly as possible over its output range.

\section{Implementation analysis and description}

During the program runtime the input data are processed in order to establish frequency counts
necessary to compute the phrase pair scores and print the complete phrase translation table.
This requires the phrase pairs to be extracted and kept in the memory along with the estimated
frequency and maximum error for Lossy Counting -- essentially, a set of triples $(e, f, \Delta)$
has to be maintained.\footnote{I.e. the data structure $D$ described in \Cref{chap:lossy-counting}.}
These data account for most of the memory required during the runtime, therefore their effective
representation in memory is mission-critical and was object of the major programming effort.

\subsection{Size of integer types}

Both frequency count and maximum error are plain unsigned integers, so at first it might seem
there is nothing that can be possibly done about their optimization.
However, it will be unwise to just declare them both as \verb|unsigned int|.
First, the size of integer data types is implementation dependent.
Despite that \verb|int| is typically 4 bytes long, this is not guaranteed \citep[Chapter 4.6]{stroustrup:cplusplus}.
Second, it is important to realize that by processing real life data we may never need as much
space as some integer types provide and saving even few bytes per item can lead to gigabytes
saved in total.

To ensure cross-platform compatibility, we base the definitions of estimated frequency count and
maximum error (and other crucial numeric types) on definitions of fixed-size integer types from
Boost.Integer library.\footurl{http://www.boost.org/doc/libs/1_53_0/libs/integer/doc/html/index.html}
For frequency counts we dedicated 4~bytes, to be able to store all unsigned values up to
$2^{32} \approx 4.3 \times 10^9$ -- this is definitely enough for our task as there is no
parallel corpus yet with billions of parallel sentences, therefore it is safe to expect no phrase
pair will occur billion times.
For the values of maximum error it is sufficient to reserve only 1~byte (range from 0 to 255)
-- maximum error is always less than the number of epochs in Lossy Counting and from our initial
experiments we had learned that having more than just a few epochs leads to pruning that already
hurts too much.\footnote{As results in \Cref{chap:results} demonstrate, with more data it is
possible to prune more (with more epochs) with no significant loss of quality, but still the 0-255
range displays a tremendous reserve.}

\subsection{Words, phrases, alignments}

% Types of objects stored:
% - words, phrases, alignments

In contrast to counters, a phrase pair is quite a complex structure: it consist of source phrase,
target phrase and also the information about alignment between particular words of both phrases.
Both source and target phrases can be further decomposed into sequence of words, while alignment
information can be further decomposed into sequence of pairs of alignment points.

Let us try to sketch a possible implementation of phrase pair structure.
Given the description above and being familiar with the containers from standard
template library, we might come up with the following declaration:
\begin{verbatim}
struct PhrasePair {
  std::vector<std::string> source; // Sequence of source words.
  std::vector<std::string> target; // Sequence of target words.
  std::vector<size_t> alignment; // Sequence of alignment points.
};
\end{verbatim}

Obviously, this is very simple and very naive solution.
The flexibility of STL containers comes at a cost: each container has to keep track of its
current size, the memory it has reserved etc.
In case of our \verb|PhrasePair| structure, the overhead of 3 instances of STL \verb|vector|
container requires 72 bytes of memory.\footnote{With GCC on 64-bit machine, on 32-bit machine
this value will be usually half as big.}
Assuming a parallel corpus with hundred of millions of phrase pairs, only to account for
this overhead, several gigabytes of memory would be required.

The main point of the hypothetical example above was therefore to stress the importance of
a careful analysis of the problem and a proper usage of optimization options and techniques
of C++ that were described in previous section.

So, let us now describe the actual implementation of phrase pairs handling within \eppex{}.
As mentioned, the main building blocks of phrase pair are words, phrases and alignments.

\subsubsection*{Words}

In \Sref{sec:strings} we have already discussed the advantage of C-strings over STL strings
when it comes to memory requirements of both.
Later, in \Sref{sec:objects-unification} we presented a technique of lowering program memory
footprint by keeping all unique copies of some type of objects in a single place and keeping
only a lightweight references to such copies in data structures that have to access them.
With both these options in mind, it is easy to come up with an idea to store all words as
C-strings in a separate storage and keep only pointers to them in phrase pair structure.
The only problem with this idea is that a pointer type in 64-bits environments occupies
8~bytes, what is simply excessive for representation of a single word (many of words could
be fit in 8 bytes directly).

Instead, as already suggested in \Sref{sec:objects-unification}, it is more effective to
use numeric index as a reference.
Although such solution incurs an additional overhead, because pointers to C-strings have to
be maintained in some index-like structure anyway, the nature of phrase pairs extraction
process suggests that each word occurrence in the input corpus will be referenced from within
several phrase pairs.\footnote{Recall that the phrase extraction algorithm may, in theory,
extract as many as $N^2$ phrases from a sentence of $N$ words.}
The data type of numeric index has to be capable of representing the whole domain of input
factor forms, therefore we picked up \verb|boost::uint32_t| data type from Boost.Integer
library for this purpose with a reasonable assumption that no model will have factor with
more than 4.2~billions of distinct forms.

\subsubsection*{Phrases}

With the words represented as numeric indexes, it is straight-forward to represent phrase as
a sequence of such indexes and this is in fact the solution we sticked to.

An alternative implementation could also make use of objects unification, but the positive
effect of this technique is doubtful on this level, because there are far more distinct
phrase forms than there are word forms.
Also, because a lot of phrase pairs will be pruned out during a typical \eppex{} run,
with object unification it would be impossible to wipe out their data without some reference
counting scheme, but such scheme would bring in an additional overhead, further diminishing
the already uncertain efficiency of such solution.\footnote{Although the same reasoning applies
for words, the scale of the problem is less prominent in this case as there are fewer word forms.
Consequently, we do not attempt to detect and remove words that are no more referenced by
a phrase pair (as a result of pruning).}

\subsubsection*{Alignments}

% Properties of phrase pair alignment.
The phrase pair alignment marks words from source and target side that were aligned to
each other in word alignment of sentence from which the phrase pair was extracted.
The phrase extraction algorithm ensures that each extracted phrase pair is based on
at least one alignment pair, ...

% TODO: Alignment length.

\subsubsection*{Phrase pairs}

Now it is time to put all the phrase pairs building blocks together: ...

\subsection{Persistence of objects}

...

% Indexed strings storage (and memory pools)

\subsection{Compressed I/O}

\Eppex{} can read/write directly from/to gzipped files, the same way legacy \emph{phrase-extract}
tools does.
This option allows to save a significant amount of disk space, as a typical phrase table will be
several times smaller when compressed.\footnote{In case of phrase tables produced in our experiments
the compression ratio of gzipped phrase tables varied between 14.0\% and 22.9\%.}
Moreover, in environments when disks are under heavy load (shared computation servers are often
the case), it may even speed up the whole I/O process.

Our implementation simply reuses the respective library shipped with Moses source code,
as it is cleanly designed and easy to include.

% "Copying" phrases between extraction and scoring


% Experiments
% In this chapter:
% - ...

\chapter{Experiments}
\label{chap:experiments}

We conducted a series of experimental phrase table extractions with both eppex and
standard phrase extraction toolkit shipped with Moses.
All runs were carefully benchmarked -- we gauged CPU time, wall-clock time and
physical memory usage of all relevant steps.
We also performed comparison of ultimate translation quality as represented by BLEU score.
In this chapter we are going to describe the details of our experiments and
the parameters of various scenarios,
while the benchmarking results are presented in the chapter that follows.

\section{Baseline}

The training process of Moses takes place in nine steps.\footurl{http://www.statmt.org/moses/?n=FactoredTraining.HomePage}
These steps cover the whole training pipeline including word alignment, lexical table construction,
phrase table construction and more. The phrase table construction itself is done in two steps,
phrase extraction and phrase scoring, which might be even further split into following
substeps: (1) phrase extraction that produces direct and reverse phrase table halves
(without scores yet); (2) sorting and (3) scoring of the direct table; (4)
sorting and (5) scoring of the reverse table; (6) sorting of the scored
reverse table; (7) consolidation of the scored direct and reverse tables.

% Introduce --parallel and --cores options.
The training process in Moses can utilize multicore architecture if it is available.

For clarity of results we sticked to recent Moses release 1.0.

\section{Benchmarking}
% What has been benchmarked:
% - CPU time, wall clock time, RAM usage
% - BLEU score
% How the benchmarking was implemented?


\section{Data}
% Data we used:
% - CzEng
% - Giga corpus

We picked up two datasets with significant amount of data: CzEng and Giga corpus.

\section{Environment}
% Few lines about the OS and HW settings of our machine.


% Experiments
\chapter{Results}
\label{chap:results}

\setlength{\epigraphwidth}{1.0\textwidth}
\epigraph{The closer a machine translation is to a professional human translation, the better it is.}{--- Papineni et al., BLEU: a Method for Automatic Evaluation of Machine Translation}

\section{Cs-En dataset}
\label{sec:cs-en-results}

We perform 14 experimental phrase table extractions for Cs-En dataset,
the comprehensive list is presented by \Tref{cs-en-wmt13-scenarios}.
We included one more optimized baseline with 8 cores, because the difference in
wall clock time necessary to construct the phrase table between runs with 4 and 8 cores
was considerable.
Sorting processes in both optimized baselines were given 18~GB of memory, so the virtual
peak of entire pipeline reached approximately the same value as in case of \eppex{}
run with no pruning.

% Cs-En: description and parameters of experiments
\begin{table}[ht]
\centering
\begin{tabular}{ r p{10cm} }
name & description and parameters \\
\hline
\hline
def-base        & Standard Moses pipeline with no special parameters \\
multi-base      & Standard Moses pipeline with \verb|--cores 4| \\
comp-base       & Standard Moses pipeline with \verb|--sort-compress gzip| and \verb|--cores 4| \\
opt-base        & Standard Moses pipeline with \verb|--sort-buffer 18G|, \verb|--sort-compress gzip| and \verb|--cores 4| \\
opt-c8-base     & Standard Moses pipeline with \verb|--sort-buffer 18G|, \verb|--sort-compress gzip| and \verb|--cores 8| \\
\hline
eppex zero      & \eppex{} set to no pruning and \verb|--GZOutput| option \\
eppex def.      & \eppex{} with \verb|--limits| set to \verb|1-3:0:1,4-5:1:4,6-7:4:8| and \verb|--GZOutput| option \\
eppex 0:n       & \eppex{} with \verb|--limits| set to \verb|1:0:1,2:0:2,...,7:0:7| and \verb|--GZOutput| option \\
eppex 0:n+1     & \eppex{} with \verb|--limits| set to \verb|1:0:2,2:0:3,...,7:0:8| and \verb|--GZOutput| option \\
eppex 1:n+1     & \eppex{} with \verb|--limits| set to \verb|1:1:2,2:1:3,...,7:1:8| and \verb|--GZOutput| option \\
\hline
sigfilter a-e   & baseline followed by significance filtering with pruning threshold $\alpha - \epsilon$ \\
sigfilter a+e   & baseline followed by significance filtering with pruning threshold $\alpha + \epsilon$ \\
sigfilter 30 a+e  & baseline followed by significance filtering with cutoff limit of 30 and pruning threshold $\alpha + \epsilon$ \\
sigfilter 30    & baseline followed by significance filtering with cutoff limit of 30 \\
\hline
\hline
\end{tabular}
\caption{\label{cs-en-wmt13-scenarios}
List of various experiments and their settings for Cs-En setup, "eppex~def." is shortcut for \emph{eppex defensive}.}
\end{table}

The \emph{eppex zero} experiment is performed to establish the maximum memory demands of \eppex{}
for the input data of a given size.
The \emph{eppex defensive} experiment has the positive and negative limits set to achieve the same
level of pruning as the \emph{eppex 1-in} experiment that performed very well in 2011 experiments.
The remaining \eppex{} experiments have a separate Lossy Counter set for every phrase length:
the negative limit is the same for all instances, but the positive limit is set in correspondence
to a phrase length to allow for a bigger estimation errors in the case of longer phrase pairs.

\subsection{Translation phrase table size and quality}

\Tref{cs-en-wmt13-pt-size-and-bleu} presents the phrase table sizes and BLEU scores for all
distinct phrase tables created in our experiments.
Despite that the phrase table sizes are very different, with the most pruned phrase table being
only $1/13$ size of the baseline, the achieved BLEU scores are very tight and the difference
between the best and the worst score is only 0.66~point in the case of \emph{wmt-12} test data
and 0.61~point in the case of \emph{wmt-13} test data.

% Cs-En: phrase tables sizes and BLEU scores
\begin{table}[ht]
\centering
\begin{tabular}{ | c | c c | c c | }
\hline
 & \multicolumn{2}{|c|}{final phrase table size} & \multicolumn{2}{|c|}{BLEU score} \\
experiment & phrase pairs & .gz file size & wmt-12 & wmt-13 \\
\hline
\hline
baseline          & 336.0~M & 8.8~GB & 0.2327 & 0.2583 \\
sigfilter 30      & 301.9~M & 8.2~GB & 0.2312 & 0.2562 \\
sigfilter a-e     & 203.1~M & 5.9~GB & 0.2301 & 0.2560 \\
eppex def.        & 109.8~M & 2.7~GB & \textbf{0.2338} & \textbf{0.2598} \\
eppex 0:n         &  86.2~M & 2.3~GB & 0.2324 & 0.2586 \\
sigfilter a+e     &  70.0~M & 1.9~GB & 0.2321 & 0.2559 \\
eppex 0:n+1       &  67.9~M & 1.8~GB & 0.2304 & 0.2561 \\
sigfilter 30 a+e  &  60.2~M & 1.7~GB & 0.2305 & 0.2563 \\
eppex 1:n+1       &  25.7~M & 0.7~GB & 0.2272 & 0.2537 \\
\hline
\end{tabular}
\caption{\label{cs-en-wmt13-pt-size-and-bleu}
Phrase table sizes and BLEU scores for the various experiments of Cs-En setup.}
\end{table}

The \emph{eppex defensive} experiment again proved to be very competitive:
the phrase table has only $1/3$ of size of the baseline,
but the BLEU score is actually better for the both test sets (although not significantly).
On the other hand, pruning of all single occurring phrase pairs in the \emph{eppex 1:n+1}
experiment resulted in the worst score on both sets, and seems to be already
too harsh for the Cs-En setup.

Interestingly, there is no clear differentiation of what was the best \emph{sigfilter}
configuration: especially, in the case of \emph{wmt-13} test data all the scores
happen to occur within a tiny range of $[0.2559, 0.2563]$, despite the phrase table sizes
differs significantly, from 60.2~M to 301.9~M of phrase pairs.

\subsection{Memory and time requirements}

\Tref{cs-en-wmt13-time-benchmarks} presents the amount of time necessary to finish
phrase table extraction with various systems and their configurations.

% Cs-En: baseline and eppex wall clock and CPU time values
\begin{table}[ht]
\centering
\begin{tabular}{ | c | r r | r r | r r | }
\hline
 & \multicolumn{2}{|c|}{total time} & \multicolumn{2}{|c|}{extraction} & \multicolumn{2}{|c|}{scoring} \\
experiment & wall & CPU & wall & CPU & wall & CPU \\
\hline
\hline
def-base      & 15.6 & 15.5 & 7.1 & 6.5 & 8.5 & 9.0 \\
multi-base    & 10.4 & 19.4 & 4.9 & 6.7 & 5.5 & 12.7 \\
comp-base     & 10.6 & 25.4 & 4.1 & 11.3 & 6.4 & 14.1 \\
opt-base      & 8.3 & 19.9 & 2.6 & 7.0 & 5.6 & 13.0 \\
opt-c8-base   & 7.0 & 20.7 & 2.3 & 7.2 & 4.7 & 13.6 \\
eppex zero    & 2.9 & 2.9 & -- & -- & -- & -- \\
\hline
eppex def.    & 1.6 & 1.6 & -- & -- & -- & -- \\
eppex 0:n     & 1.5 & 1.5 & -- & -- & -- & -- \\
eppex 0:n+1   & 1.4 & 1.4 & -- & -- & -- & -- \\
eppex 1:n+1   & 1.2 & 1.2 & -- & -- & -- & -- \\
\hline
\end{tabular}
\caption{\label{cs-en-wmt13-time-benchmarks}
Wallclock times and CPU usage values (in hours) of the phrase table
construction for the various experiments of Cs-En setup.}
\end{table}

Even with no pruning \eppex{} is capable of doing the phrase table construction more
than twice as fast as the most challenging baseline (and using 7 times less CPU time).
When compared to the default baseline the difference is even more pronounced with \eppex{}
being 5~times as fast.

From the comparison of the default and multi-core baseline it seems clear that whenever there
is a possibility to employ multiple cores it should be taken: just by running with 4 cores
the baseline execution time has been cut to $2/3$.
Doubling number of cores from 4 to 8 in optimized experiments resulted in further reduction
of execution time, although not as significant (by 15\%).

An important observation is that adding only the option to make \texttt{sort} program gzip
its temporary data, did not help to decrease the total execution time.
More precisely, it did help in phrase extraction step (ca. $-0.8~h$), but did the very opposite
in scoring step (ca. $+0.9~h$), so the overall impact on the total wall clock time was eventually
a bit negative (ca. $+0.2~h$).
A reasonable explanation for such, perhaps unexpected, behavior stems from the fact that in phrase
extraction step both phrase table halves are sorted simultaneously, whereas in scoring step
only indirect half is sorted:
apparently, in our environment forcing \texttt{sort} to gzip its temporary data makes it run
slower when there are no other processes accessing the disk (as in the scoring step), but reducing
the disk access by the same means when two similar sort tasks run simultaneously can benefit
in both of them finishing faster (as in the phrase extraction step).
This explanation might be further supported by the fact that a lot of \texttt{sort} temporary
data gzipping took place in \emph{comp-base} experiment, because the striking increase of CPU time
consumption can only be attributed to this extra gzipping: % TODO: Reformulate, unreadable.
phrase extraction required almost 4.5 CPU hours more, scoring required almost 1.5 CPU hours
more.\footnote{Given that there are two and one sorting processes, one could expect the ratio
4.5:1.5 of additional CPU time to be closer to 2:1, but after phrase extraction phrase table halves
contain every phrase pair occurrence, whereas after scoring they contain only unique phrase pairs,
so the size of data to sort is rather close to 3:1 than 2:1 ratio.}
The similar conclusions might be made also by examination of partial wall clock times of \emph{multi-base}
and \emph{opt-base} experiments: while the phrase extraction time fell down from 4.9~hours to 2.6~hours
(i.e. almost to $1/2$), the scoring time jumped up a bit from 5.5~hours to 5.6~hours.
Thus, even the significantly increased sorting buffer managed only to neutralize the negative impact of
temporary data compression on wall clock time of scoring step, but in the same time it further pronounced
the positive impact of the same settings on phrase extraction step.

However, to put things in a wider perspective, we believe that the option to compress temporary data of
\texttt{sort} routine has been introduced in order to limit the peak usage of disk space, which may
otherwise pose a limiting factor in some environments, as the figures presented in the following text
demonstrate.

\Tref{cs-en-wmt13-vm-and-disk-usage-peaks} illustrates both memory and disk space demands of the baseline
and \eppex{} experiments.

% Cs-En: virtual memory and disk usage peaks
\begin{table}[ht]
\centering
\begin{tabular}{ | c | r l | r | }
\hline
 & \multicolumn{2}{|c|}{VM peak} & \\
experiment & size & step & du peak \\
\hline
\hline
def-base       &  2.0~GB &    scoring & 206.7~GB \\
multi-base     &  6.2~GB &    scoring & 213.5~GB \\
comp-base      &  6.2~GB &    scoring &  43.4~GB \\
opt-base       & 36.2~GB & extraction &  42.8~GB \\
opt-c8-base    & 36.2~GB & extraction &  42.8~GB \\
eppex zero     & 36.8~GB &      eppex &   8.8~GB \\
\hline
eppex def.     & 18.1~GB &      eppex &   2.7~GB \\
eppex 0:n      & 10.5~GB &      eppex &   2.2~GB \\
eppex 0:n+1    &  8.4~GB &      eppex &   1.8~GB \\
eppex 1:n+1    &  9.6~GB &      eppex &   0.7~GB \\
\hline
\end{tabular}
\caption{\label{cs-en-wmt13-vm-and-disk-usage-peaks}
Virtual memory and disk usage peaks of the phrase table construction for various experiments of "Cs-En" setup.}
\end{table}

The default baseline experiment had the lowest memory demands from all experiments, it needed only 2~GB
of memory to finish.
With non-default settings the memory consumption tripled when multiple cores were utilized (in \emph{multi-base}
and \emph{comp-base} experiments), except for cases when it was dominated by the size of buffer dedicated
to \texttt{sort} routine (in optimized experiments).

\Eppex{} memory demands are largely dependent on the amount of pruning requested: with \emph{zero}
pruning almost 37~GB of memory was consumed, mild \emph{defensive} pruning required only half as much
memory and harsh \emph{1:n+1} pruning only $1/4$ (but still almost 10~GB).
The comparison of VM peaks and phrase table sizes of \emph{eppex 0:n+1} and \emph{eppex 1:n+1}
experiments confirms the findings from \Sref{sec:lossy-counting-applicability}:
because of the harsher limits, \emph{eppex 1:n+1} produced a half as big phrase table as \emph{eppex 0:n+1},
but since it had a smaller estimation error, it required 15\% more memory.

The disk usage peak exceeded 200~GB in the baseline experiments without compression of
\texttt{sort} temporary data, with the compression it lowered to approximately 40~GB
(notably still above the memory peak of \eppex{} with no pruning).
As noted before, \eppex{} produces no temporary data, therefore in all experiments the
disk usage peaks strictly copied the size of the produced phrase tables.

Finally, \Tref{cs-en-wmt13-sigfilter-runtime-benchmarks} presents the runtime benchmarking
values for different runs of significance filtering.
Even the simple histogram pruning required more time than any \eppex{} run with nonzero
pruning limits.\footnote{This is partially due a design flaw in the \texttt{sigfilter} tool:
the SALM-indexed corpus is always required to proceed, even though it is not necessary in
the case of histogram pruning. The unnecessary SALM indexing accounts for a large part
of the reported wall clock time.}
On the other hand, the memory demands of \emph{sigfilter}, although considerable, were
always below those of \eppex{}.

% Cs-En: sigfilter runtime requirements
\begin{table}[ht]
\centering
\begin{tabular}{ | c | r r | r l | }
\hline
 & \multicolumn{2}{|c|}{time} & \multicolumn{2}{|c|}{VM peak} \\
experiment & wall & CPU & size & step \\
\hline
\hline
sigfilter a-e     & 10.2~h & 11.3~h & 6.7~GB & filtering \\
sigfilter a+e     & 9.9~h & 10.6~h & 6.8~GB & filtering \\
sigfilter 30 a+e  & 8.1~h & 8.7~h & 6.7~GB & filtering \\
sigfilter 30      & 2.5~h & 3.4~h & 2.7~GB & indexing \\
\hline
\end{tabular}
\caption{\label{cs-en-wmt13-sigfilter-runtime-benchmarks}
The complete benchmarking figures for various settings of \emph{sigfilter} in Cs-En setup:
wallclock time, CPU usage value and virtual memory peak with the step, in which it occurred.}
\end{table}

\section{Fr-En dataset}
\label{sec:fr-en-results}

\Tref{fr-en-80-scenarios} presents the list of all the experimental phrase table extraction
performed on the Fr-En dataset.

% Fr-En: description and parameters of experiments
\begin{table}[ht]
\centering
\begin{tabular}{ r p{10cm} }
name & description and parameters \\
\hline
\hline
opt-base      & Standard Moses pipeline with \verb|--sort-buffer 100G|, \verb|--sort-compress gzip| and \verb|--cores 4| \\
opt-c8-base   & Standard Moses pipeline with \verb|--sort-buffer 100G|, \verb|--sort-compress gzip| and \verb|--cores 8| \\
eppex zero    & \eppex{} set to no pruning and \verb|--GZOutput| option \\
eppex def.    & \eppex{} with \verb|--limits| set to \verb|1-3:0:1,4-5:1:4,6-7:4:8| and \verb|--GZOutput| option \\
eppex zero-n  & \eppex{} with \verb|--limits| set to \verb|1:0:1,2:0:2,...,7:0:7| and \verb|--GZOutput| option \\
eppex 1:n+1   & \eppex{} with \verb|--limits| set to \verb|1:1:2,2:1:3,...,7:1:8| and \verb|--GZOutput| option \\
eppex 2:n+2   & \eppex{} with \verb|--limits| set to \verb|1:2:3,2:2:4,...,7:2:9| and \verb|--GZOutput| option \\
\hline
\hline
\end{tabular}
\caption{\label{fr-en-80-scenarios}
List of various experiments and their settings for "Fr-En" setup.}
\end{table}

In the case of baseline experiments, we had attempted all the configurations mentioned
in \Sref{sec:baseline-experiments}, but managed to only finish the optimized ones.
The non-optimized attempts usually died in the middle of sorting with \texttt{sort} complaining
about missing temporary files.
We did not investigate the exact reason, but our raw guess is that the number of temporary
files reached some system-imposed limit, as there were thousands of temporary files left in
the working directory after the process died.

The evaluated \eppex{} configurations are basically the same as in the case of Cs-En setup,
we only included an additional, harsher configuration \emph{eppex 2:n+2}.

\subsection{Translation phrase table size and quality}

\Tref{fr-en-pt-size-and-bleu} presents phrase table sizes and BLEU scores for all
distinct phrase tables created in our experiments.

% Fr-En: phrase tables sizes and BLEU scores
\begin{table}[ht]
\centering
\begin{tabular}{ | c | r r | c c | }
\hline
 & \multicolumn{2}{|c|}{final phrase table size} & \multicolumn{2}{|c|}{BLEU score} \\
experiment & phrase pairs & .gz file size & wmt-12 & wmt-13 \\
\hline
\hline
baseline          & 1782.0~M & 42.5~GB & 0.2860 & 0.2958 \\
sigfilter 30      & 1470.8~M & 35.9~GB & 0.2849 & 0.2964 \\
sigfilter a-e     &  811.4~M & 21.0~GB & 0.2845 & 0.2953 \\
eppex def.        &  463.4~M & 10.6~GB & \textbf{0.2867} & 0.2950 \\
sigfilter a+e     &  329.5~M &  8.4~GB & 0.2866 & \textbf{0.2977} \\
eppex zero-n      &  283.2~M &  7.0~GB & 0.2841 & 0.2950 \\
sigfilter 30 a+e  &  270.0~M &  7.0~GB & 0.2857 & 0.2976 \\
eppex 1:n+1       &  127.9~M &  3.2~GB & 0.2859 & 0.2953 \\
eppex 2:n+2       &   77.0~M &  2.0~GB & 0.2839 & 0.2943 \\
\hline
\end{tabular}
\caption{\label{fr-en-pt-size-and-bleu}
Phrase table sizes and BLEU scores for various experiments of "Fr-En" setup.}
\end{table}

...

\subsection{Memory and time requirements}

\Tref{fr-en-time-benchmarks} presents the amount of time necessary to finish
phrase table extraction with various systems and their configurations.

% Fr-En: baseline and eppex wall clock and CPU time values
\begin{table}[ht]
\centering
\begin{tabular}{ | c | r r | r r | r r | }
\hline
 & \multicolumn{2}{|c|}{total time} & \multicolumn{2}{|c|}{extraction} & \multicolumn{2}{|c|}{scoring} \\
experiment & wall & CPU & wall & CPU & wall & CPU \\
\hline
\hline
opt-base      & 56.8 & 146.1 & 15.0 & 39.2 & 41.8 & 106.9 \\
opt-c8-base   & 45.1 & 149.8 & 12.7 & 39.7 & 32.4 & 110.1 \\
eppex zero    & 25.5 &  25.4 & -- & -- & -- & -- \\
\hline
eppex def.    & 12.2 & 12.2 & -- & -- & -- & -- \\
eppex zero-n  & 10.7 & 10.7 & -- & -- & -- & -- \\
eppex 1:n+1   & 9.5 & 9.5 & -- & -- & -- & -- \\
eppex 2:n+2   & 9.3 & 9.2 & -- & -- & -- & -- \\
\hline
\end{tabular}
\caption{\label{fr-en-time-benchmarks}
Wallclock times and CPU usage values (in hours) of the phrase table
construction for various experiments of "Fr-En" setup.}
\end{table}

...

\Tref{fr-en-vm-and-disk-usage-peaks} illustrates both memory and disk space demands of baseline
and \eppex{} experiments.

% Fr-En: virtual memory and disk usage peaks
\begin{table}[ht]
\centering
\begin{tabular}{ | c | r l | r | }
\hline
 & \multicolumn{2}{|c|}{VM peak} & \\
experiment & size & step & du peak \\
\hline
\hline
opt-base       & 200.1~GB & extraction & 160.5~GB \\
opt-c8-base    & 200.1~GB & extraction & 160.5~GB \\
eppex zero     & 203.0~GB &      eppex &  48.0~GB \\
\hline
eppex def.     &  89.9~GB &      eppex &  16.0~GB \\
eppex zero-n   &  48.7~GB &      eppex &   8.0~GB \\
eppex 1:n+1    &  46.4~GB &      eppex &   4.0~GB \\
eppex 2:n+2    &  45.6~GB &      eppex &   2.0~GB \\
\hline
\end{tabular}
\caption{\label{fr-en-vm-and-disk-usage-peaks}
Virtual memory and disk usage peaks of the phrase table construction for various experiments of "Fr-En" setup.}
\end{table}

...

Finally, \Tref{fr-en-sigfilter-runtime-benchmarks} presents the runtime benchmarking
values for different runs of significance filtering.\footnote{Due a tight experiments
schedule we run the significance filtering experiments on Fr-En setup on the set of
machines designated for Cs-En setup.}

% TODO: Any comment?

% Fr-En: sigfilter runtime requirements
\begin{table}[ht]
\centering
\begin{tabular}{ | c | r r | r l | }
\hline
 & \multicolumn{2}{|c|}{time} & \multicolumn{2}{|c|}{VM peak} \\
experiment & wall & CPU & size & step \\
\hline
\hline
sigfilter a-e     & 82.7~h & 87.9~h & 28.2~GB & filtering \\
sigfilter a+e     & 83.9~h & 87.2~h & 28.5~GB & filtering \\
sigfilter 30 a+e  & 46.9~h & 50.3~h & 28.4~GB & filtering \\
sigfilter 30      & 10.3~h & 13.8~h & 11.5~GB & indexing \\
\hline
\end{tabular}
\caption{\label{fr-en-sigfilter-runtime-benchmarks}
Wallclock times and CPU usage values, virtual memory peak value and the step,
in which it occurred, of significance filtering for various settings of \emph{sigfilter} in "Fr-En" setup.}
\end{table}

\section{Comparison to 2011 version}
\label{sec:cu-bojar-results}
% Comparison of current eppex and phrase-extract performance vs. mid-2011 state

\citet{przywara:eppex} presented an evaluation of an early version of \eppex{}
that was able to only extract phrases. It performed a faster phrase
table creation than the legacy \emph{phrase-extract} in such situations,
where lossy counting resulted in significant filtration of extracted phrase
pairs and consequent sub-steps (sorting, scoring and consolidation of both
phrase table halves) had to process reduced amount of data;
eventually the whole process of phrase table creation finished faster.

As we had all the parallel data used in their experiments available and
we perform our runtime benchmarking in almost identical manner,
we decided to carry out comparison between the early and current versions of
both \eppex{} as well as \emph{phrase-extract} toolkit.

\subsection{Implementation differences}

The current version of \eppex{} differs from the early version mainly in the
capability of performing not only phrase pairs extraction, but a complete
phrase table construction (that is including phrase pairs scoring).
The early version, however, could be used to extract orientation info
the same way \emph{extract} can, while this functionality has been dropped
from the current version as not related to the core task of phrase table creation.
Following the upgrade of \emph{phrase-extract} suite, an option to read from
gzipped input files and dump gzipped output files has been added to \eppex{}.
Internally, there have been a multitude of performance tweaks,
as performance optimization is the most important aim in \eppex{} development,
but a full listing of implementation changes would be purposeless here.

On the contrary, \emph{phrase-extract} suite has been updated mainly in order to
provide a richer functionality, but three changes since the mid-2011 version
had a sound impact on its runtime performance: the optimization of target phrases
scanning in the \emph{scorer} implementation,\footnote{Commit 677378774aca30c8f0d4ca57267f7ac5ef7d7cb6.}
adding an option to gzip the output directly within the main three binaries
(\emph{extract}, \emph{scorer} and \emph{consolidate})
and the parallelization of both phrase extraction and phrase scoring
steps.\footnote{See \texttt{extract-parallel.perl} and \texttt{score-parallel.perl}
in \texttt{<moses>/scripts/generic/}.}

\subsection{Parameters of experiments}

\Tref{cu-bojar-scenarios} presents all the experiments and their settings.
Again, we have included one more optimized baseline with 7 cores, because
the achieved speeding up is considerable.
The \emph{default baseline} is invoked with the same parameters as the \emph{2011
baseline}, except for an option that turns on gzipping of temporary data and
phrase table that is now implicitly activated from within the training script.
Because of this, we also turn \verb|--GZOutput| on for all \eppex{} runs.
We experimented with different pruning parameters: \emph{eppex-1-in} (milder
pruning) and \emph{eppex-1-out} (harsher pruning) are using the same pruning
parameters as the 2011 experiments and we also performed \emph{eppex zero}
experiment (an \eppex{} run with no pruning).

% Detailed info of cu-bojar experiments
\begin{table}[ht]
\centering
\begin{tabular}{ r p{10cm} }
name & description and parameters \\
\hline
\hline
def-base        & Standard Moses pipeline with no special parameters \\
multi-base      & Standard Moses pipeline with \verb|--cores 4| \\
comp-base       & Standard Moses pipeline with \verb|--sort-compress gzip|
  and \verb|--cores 4| \\
opt-base        & Standard Moses pipeline with \verb|--sort-buffer 12G|,
  \verb|--sort-compress gzip| and \verb|--cores 4| \\
opt-c7-base     & Standard Moses pipeline with \verb|--sort-buffer 12G|,
  \verb|--sort-compress gzip| and \verb|--cores 7| \\
eppex zero      & \eppex{} set to no pruning and \verb|--GZOutput| option \\
eppex 1-in      & \eppex{} with pruning thresholds set to keep in
  all phrase pairs of length 1--3 and prune longer phrase pairs
  with max. positive threshold of 8 and \verb|--GZOutput| option \\
eppex 1-out     & \eppex{} with pruning thresholds set to remove
  all single-occurring phrase pairs and prune the rest with
  max. positive threshold of 8 and \verb|--GZOutput| option \\
\hline
\hline
\end{tabular}
\caption{\label{cu-bojar-scenarios}List of various experiments and their
settings for "cu-bojar" setup. The parameters of "eppex 1-in" and "eppex 1-out"
conform to the corresponding 2011 experiments, "def-base" corresponds to the 2011
"baseline" experiment except for (now activated implicitly) gzipping of
temporary data.}
\end{table}

\subsection{Memory and time requirements}

% NOTE:
% Steps and substeps of phrase table construction via train-model.perl back in 2011:
% (1) Phrase extraction
%  a) extract
%  b) gzip f2e
%  c) gzip e2f
% (2) Phrase scoring
%  d) sort f2e (now done in (1) and so reported as part of phrase extraction in table below)
%  e) score e2f
%  f) sort e2f (now done in (1) and so reported as part of phrase extraction in table below)
%  g) score e2f
%  h) sort inv
%  i) cons
%  j) gzip pt

\Tref{cu-bojar-time-benchmarks} compares wallclock times and CPU usage of all the
experiments and in case of baseline and 2011 experiments also separately for phrase
extraction and phrase scoring steps.

The comparison between the old and current baselines reveals that the optimization of
\emph{scorer} mentioned above resulted in a speed up by more than a factor of two.
On the other hand the default phrase extraction became slightly more time demanding,
but this may be explained by the fact that parallelization incurred some overhead
that does not pay back when running only with a single core, but significantly cuts
down the running time when using multiple cores: in our setup using 7~cores (along with
more memory for sorting) lowered time of phrase extraction to 40\% and of phrase scoring
to almost 50\% of respective default measures.

The comparison between 2011 and current version of \eppex{} also displays a significant
speed up: with harsh pruning the phrase table construction is done in half an hour (took
almost two hours in 2011) and without any pruning the full phrase table was built in
less than hour and half, thus \eppex{} proves to be a viable alternative to
\emph{phrase-extract} also in situations, when pruning is not an option (being twice as
fast than the optimized baseline).

\begin{table}[ht]
\centering
\begin{tabular}{ | c | r r | r r | r r | }
\hline
 & \multicolumn{2}{|c|}{total time} & \multicolumn{2}{|c|}{extraction} & \multicolumn{2}{|c|}{scoring} \\
experiment & wall & CPU & wall & CPU & wall & CPU \\
\hline
\hline
baseline*     & 8.8 & 7.1 & 2.1 & 1.0 & 6.7 & 6.1 \\
def-base      & 5.6 & 4.8 & 2.5 & 1.5 & 3.1 & 3.2 \\
multi-base    & 4.3 & 5.7 & 2.2 & 1.5 & 2.1 & 4.1 \\
comp-base     & 3.3 & 7.1 & 1.1 & 2.5 & 2.3 & 4.6 \\
opt-base      & 3.2 & 6.0 & 1.1 & 1.7 & 2.1 & 4.3 \\
opt-c7-base   & 2.8 & 6.0 & 1.1 & 1.6 & 1.7 & 4.3 \\
eppex zero    & 1.4 & 1.4 & -- & -- & -- & -- \\
\hline
eppex 1-in*   & 4.1 & 3.7 & 1.8 & 1.5 & 2.2 & 2.2 \\
eppex 1-in    & 0.8 & 0.8 & -- & -- & -- & -- \\
\hline
eppex 1-out*  & 1.8 & 1.6 & 1.6 & 1.4 & 0.2 & 0.2 \\
eppex 1-out   & 0.5 & 0.5 & -- & -- & -- & -- \\
\hline
\end{tabular}
\caption{\label{cu-bojar-time-benchmarks}
Wallclock times and CPU usage values (in hours) of the phrase table
construction for various experiments of "cu-bojar" setup.
Asterisk denotes the particular measures from 2011 experiments.}
\end{table}

\Tref{cu-bojar-vm-and-du-peaks} presents memory and disk usage peaks of all the experiments
(except for disk usage peaks that are not available for 2011 experiments).

Memory demands of the baseline remain low, the 0.1~GB difference between 2011 and now
stems from the fact that our benchmarking script includes the main process into the set of
measured processes.
An optimized baseline is the most memory demanding, but this is only because we assigned 12~GB
of RAM to the sorting processes and both phrase table halves are sorted simultaneously after
phrase extraction.

Memory demands of the current version of \eppex{} are significantly lower than of the 2011 version:
in 2011, the harsh pruning setup required as much memory as the current version run without
any pruning at all.

The disk peak usage is significantly reduced in \emph{compressed} baseline, but rises almost to
default levels in optimized baselines, despite the \verb|--sort-compress| option has been applied
to them as well.
We do not poses sufficient knowledge of the inner workings of \emph{GNU sort}, therefore we do not
attempt to give an explanation of this peculiar behavior.

\begin{table}[ht]
\centering
\begin{tabular}{ | c | r | r | }
\hline
experiment & VM peak & du peak \\
\hline
\hline
baseline*     &  1.1~GB &      -- \\
def-base      &  1.2~GB & 51.8~GB \\
multi-base    &  3.9~GB & 53.6~GB \\
comp-base     &  3.9~GB & 14.3~GB \\
opt-base      & 24.1~GB & 43.7~GB \\
opt-c7-base   & 24.1~GB & 43.7~GB \\
eppex zero    & 16.8~GB &  3.7~GB \\
\hline
eppex 1-in*   & 19.2~GB &      -- \\
eppex 1-in    & 13.4~GB &  1.3~GB \\
\hline
eppex 1-out*  & 16.7~GB &      -- \\
eppex 1-out   & 11.3~GB &  0.3~GB \\
\hline
\end{tabular}
\caption{\label{cu-bojar-vm-and-du-peaks}
Virtual memory and disk usage peaks of the phrase table construction for various experiments
of "cu-bojar" setup. Asterisk denotes the particular measures from 2011 experiments.}
\end{table}

\section{Eppex and memory demands}
\label{sec:eppex-memory-demands}

We performed an additional series of experimental phrase table extractions with
the Cs-En and Fr-En datasets to get an insight on the impact of training data size and
values of pruning limits on memory demands of \eppex{} and the amount of phrase pairs
extracted.

For both datasets we created a list of input size cuts and then with each cut of size $K$
we did the following:
\begin{enumerate}
  \item From the first $K$ sentences of the whole dataset and their word alignments, we
    created a temporary training dataset.
  \item Using Moses \texttt{train-model.perl} script, we constructed lexical scores tables
    from the temporary training data.\footurl{http://www.statmt.org/moses/?n=FactoredTraining.GetLexicalTranslationTable}
  \item Finally, we used the temporary training data and lexical scores tables to perform several epochal
    extractions with some of the configurations mentioned in
    \Sref{sec:cs-en-results} (for Cs-En cuts) and \Sref{sec:fr-en-results} (for Fr-En cuts).
    We measured the virtual memory peak and recorded phrase table size of each extraction.
\end{enumerate}

\subsection{Fr-En dataset}

Our French-English training data contained only a single factor: a lowercased token.
There was in average 30 French and 25.5 English tokens per sentence in the whole data,
but in most of the cuts these averages were lower (down to 25.5 tokens per sentence
in case of French and 24 tokens per sentence in case of English).

\Tref{fr-en-memory-benchmarking} presents the comparison of virtual memory peaks
of several \eppex{} configurations described in \Sref{sec:fr-en-results}, while
\Tref{fr-en-output-size-benchmarking} presents corresponding phrase table sizes obtained.

% Memory benchmarking table for Fr-En
\begin{table}[ht]
\centering
\begin{tabular}{ | r | r | r | r | r | r | r | r | }
\hline
\multicolumn{3}{|c|}{input data size} & \multicolumn{5}{|c|}{virtual memory peak per eppex configuration} \\
\hline
sent. & source & target & zero & def. & zero-n & 1:n+1 & 1:n+2 \\
\hline
\hline
.1~M & 2.6~M & 2.4~M & 0.9~GB & 0.4~GB & 0.3~GB & 0.2~GB & 0.2~GB \\
.2~M & 5.1~M & 4.8~M & 1.7~GB & 0.8~GB & 0.4~GB & 0.4~GB & 0.3~GB \\
.3~M & 7.7~M & 7.2~M & 2.5~GB & 1.2~GB & 0.6~GB & 0.6~GB & 0.5~GB \\
.4~M & 10.2~M & 9.6~M & 3.2~GB & 1.5~GB & 0.8~GB & 0.8~GB & 0.6~GB \\
.5~M & 12.8~M & 12.1~M & 4.0~GB & 2.0~GB & 1.0~GB & 0.9~GB & 0.7~GB \\
.6~M & 15.5~M & 14.5~M & 5.0~GB & 2.3~GB & 1.2~GB & 1.1~GB & 0.9~GB \\
.7~M & 18.2~M & 16.9~M & 5.7~GB & 2.6~GB & 1.4~GB & 1.3~GB & 1.0~GB \\
.8~M & 21.0~M & 19.3~M & 6.4~GB & 3.0~GB & 1.6~GB & 1.4~GB & 1.2~GB \\
.9~M & 23.7~M & 21.8~M & 7.1~GB & 3.4~GB & 1.8~GB & 1.7~GB & 1.3~GB \\
1~M & 26.6~M & 24.3~M & 7.8~GB & 3.8~GB & 2.0~GB & 1.8~GB & 1.4~GB \\
\hline
2~M & 53.6~M & 48.4~M & 15.4~GB & 7.4~GB & 3.8~GB & 3.4~GB & 2.6~GB \\
3~M & 80.2~M & 72.3~M & 23.6~GB & 10.4~GB & 5.5~GB & 5.0~GB & 3.9~GB \\
4~M & 109.3~M & 98.8~M & 29.7~GB & 13.6~GB & 7.0~GB & 6.5~GB & 5.1~GB \\
5~M & 138.5~M & 125.2~M & 35.0~GB & 15.7~GB & 8.5~GB & 7.9~GB & 6.2~GB \\
6~M & 165.6~M & 148.1~M & 42.4~GB & 18.0~GB & 9.8~GB & 9.1~GB & 7.3~GB \\
7~M & 193.3~M & 171.2~M & 46.3~GB & 20.1~GB & 10.7~GB & 10.1~GB & 8.1~GB \\
8~M & 222.3~M & 195.2~M & 50.8~GB & 21.6~GB & 11.6~GB & 10.8~GB & 9.1~GB \\
9~M & 251.4~M & 219.3~M & 55.3~GB & 23.7~GB & 12.6~GB & 11.8~GB & 9.9~GB \\
10~M & 279.4~M & 242.1~M & 59.6~GB & 25.4~GB & 13.9~GB & 13.0~GB & 10.6~GB \\
\hline
12~M & 338.2~M & 291.1~M & 69.1~GB & 29.1~GB & 16.0~GB & 14.8~GB & 12.4~GB \\
14~M & 398.3~M & 341.4~M & 84.3~GB & 34.8~GB & 18.9~GB & 17.5~GB & 14.2~GB \\
16~M & 458.2~M & 391.4~M & 93.8~GB & 39.7~GB & 21.1~GB & 19.4~GB & 16.0~GB \\
18~M & 519.1~M & 442.3~M & 103~GB & 44.2~GB & 23.6~GB & 21.8~GB & 17.6~GB \\
20~M & 580.8~M & 493.9~M & 113~GB & 48.4~GB & 25.8~GB & 23.7~GB & 19.1~GB \\
\hline
25~M & 749.3~M & 636.4~M & 141~GB & 61.5~GB & 33.1~GB & 30.2~GB & 25.3~GB \\
30~M & 903.5~M & 768.7~M & 174~GB & 73.6~GB & 38.8~GB & 36.1~GB & 30.5~GB \\
35~M & 1050~M & 895.1~M & 190~GB & 85.5~GB & 44.9~GB & 42.6~GB & 35.1~GB \\
full & 1173~M & 1001~M & 203~GB & 89.9~GB & 48.7~GB & 46.4~GB & ??.?~GB \\ % TODO: Fill in missing value.
\hline
\end{tabular}
\caption{\label{fr-en-memory-benchmarking}
Virtual memory peaks of the phrase table construction performed with
various configurations of \eppex{} on portions of "Fr-En" dataset.
The input data size triple stands for: number of parallel sentences and number of words on the source and target side.
The full corpus had more than 39~M of parallel sentences.}
\end{table}

% TODO: Comment on the memory probing.
...

% Output size benchmarking table for Fr-En
\begin{table}[ht]
\centering
\begin{tabular}{ | r | r | r | r | r | r | r | r | }
\hline
\multicolumn{3}{|c|}{input data size} & \multicolumn{5}{|c|}{phrase table size per eppex configuration} \\
\hline
sent. & source & target & zero & def. & zero-n & 1:n+1 & 1:n+2 \\
\hline
\hline
.1~M & 2.6~M & 2.4~M & 7.7~M & 2.8~M & 1.6~M & 237~K & 182~K \\
.2~M & 5.1~M & 4.8~M & 14.8~M & 5.2~M & 3.0~M & 471~K & 362~K \\
.3~M & 7.7~M & 7.2~M & 22.0~M & 7.5~M & 4.4~M & 695~K & 531~K \\
.4~M & 10.2~M & 9.6~M & 29.3~M & 10.0~M & 5.9~M & 940~K & 719~K \\
.5~M & 12.8~M & 12.1~M & 36.8~M & 12.5~M & 7.4~M & 1.2~M & 891~K \\
.6~M & 15.5~M & 14.5~M & 44.1~M & 14.9~M & 8.8~M & 1.4~M & 1.1~M \\
.7~M & 18.2~M & 16.9~M & 51.2~M & 17.2~M & 10.1~M & 1.7~M & 1.3~M \\
.8~M & 21.0~M & 19.3~M & 58.3~M & 19.4~M & 11.4~M & 2.0~M & 1.5~M \\
.9~M & 23.7~M & 21.8~M & 65.3~M & 21.6~M & 12.6~M & 2.3~M & 1.7~M \\
1~M & 26.6~M & 24.3~M & 72.5~M & 23.9~M & 14.1~M & 2.5~M & 1.9~M \\
\hline
2~M & 53.6~M & 48.4~M & 142.1~M & 44.7~M & 27.1~M & 4.9~M & 3.7~M \\
3~M & 80.2~M & 72.3~M & 211.1~M & 64.7~M & 39.9~M & 7.2~M & 5.6~M \\
4~M & 109.3~M & 98.8~M & 270.3~M & 78.7~M & 44.1~M & 10.1~M & 8.0~M \\
5~M & 138.5~M & 125.2~M & 322.0~M & 88.5~M & 49.7~M & 12.7~M & 10.1~M \\
6~M & 165.6~M & 148.1~M & 367.8~M & 102.1~M & 60.4~M & 17.9~M & 15.0~M \\
7~M & 193.3~M & 171.2~M & 406.8~M & 114.7~M & 66.6~M & 25.5~M & 22.2~M \\
8~M & 222.3~M & 195.2~M & 450.6~M & 127.1~M & 81.1~M & 30.2~M & 25.7~M \\
9~M & 251.4~M & 219.3~M & 495.3~M & 141.5~M & 90.1~M & 36.5~M & 30.8~M \\
10~M & 279.4~M & 242.1~M & 538.9~M & 154.3~M & 103.6~M & 40.2~M & 34.0~M \\
\hline
12~M & 338.2~M & 291.1~M & 632.4~M & 179.3~M & 123.2~M & 44.3~M & 36.0~M \\
14~M & 398.3~M & 341.4~M & 726.8~M & 203.5~M & 141.6~M & 46.8~M & 37.3~M \\
16~M & 458.2~M & 391.4~M & 819.8~M & 225.4~M & 158.6~M & 49.3~M & 38.9~M \\
18~M & 519.1~M & 442.3~M & 913.2~M & 247.8~M & 176.5~M & 53.2~M & 41.9~M \\
20~M & 580.8~M & 493.9~M & 1007~M & 270.7~M & 195.4~M & 57.8~M & 45.5~M \\
\hline
25~M & 749.3~M & 636.4~M & 1290~M & 349.2~M & 268.0~M & 70.6~M & 55.1~M \\
30~M & 903.5~M & 768.7~M & 1502~M & 401.4~M & 269.1~M & 90.6~M & 72.9~M \\
35~M & 1050~M & 895.1~M & 1659~M & 437.9~M & 269.7~M & 110.8~M & 91.1~M \\
full & 1173~M & 1001~M & 1782~M & 463.4~M & 283.2~M & 127.9~M & ??.?~M \\ % TODO: Fill in missing value.
\hline
\end{tabular}
\caption{\label{fr-en-output-size-benchmarking}
Phrase table sizes (in phrase pairs) obtained with various configurations of \eppex{}
on portions of "Fr-En" dataset. The input data size triple stands for: number of
parallel sentences and number of words on the source and target side.
The full corpus had more than 39~M of parallel sentences.}
\end{table}

% TODO: Comment on phrase table sizes.
...


% TEMPORARY: what didn't make it into any other chapter
\chapter{Related}
% Stuff, that didn't make it into any other chapter.

% TODO: Chapter as such should not be present in the final version of work.


% Conclusions
\chapter{Conclusions}
\label{chap:conclusions}

\setlength{\epigraphwidth}{1.0\textwidth}
\epigraph{No one involved in computers would ever say that a certain amount of memory is enough for all time.}{--- Bill Gates, commenting on the perhaps most famous remark attributed to him}

In this work we examined applicability of the Lossy Counting algorithm,
that was designed to deliver approximate frequency counts over stream of
input items, to perform on-the-fly filtration of phrase pairs extracted in
the process of phrase table creation in Statistical Machine Translation systems.

To tackle this task, we implemented a software tool that builds a complete phrase
translation table with the maximum likelihood scores derived from the frequency
counts as approximated by the lossy counting of phrase pairs extracted from
parallel corpora during the process of translation model training.
Because internally the Lossy Counting algorithm splits the input stream into
epochs, we dubbed our tool \emph{eppex}, an acronym for \emph{epochal phrase
pairs extractor}.

On top of the standard instantiation of the Lossy Counting algorithm with
\emph{support} and \emph{error} thresholds, we devised a more intuitive interface
to invoke \eppex{}: the \emph{positive} and \emph{negative} limits that
directly relate to the true frequencies of the extracted phrase pairs.
We show the relation of the phrase table pruning activated with these limits
to the existing criterion of \emph{count-based pruning}.
We also discussed the impact of various settings of these limits on the memory
demands and output size of the epochal extraction: especially, we stressed out
that there is no direct relation between the degree of pruning and the amount
of memory consumed by the program.

We performed a series of the experimental phrase table extractions and carefully
benchmarked the performance of \eppex{} and of a baseline system for phrase
table construction, for which we chose the \emph{phrase-extract} toolkit from
an open-source SMT system Moses.
Moreover, in our experiments, we included also a popular tool for phrase table
pruning that is shipped with Moses, the significance filter or \emph{sigfilter}
in short.

The results we had obtained, showed, that \eppex{} was capable to prune off
substantial amounts of phrase pairs (up to 95\% of the phrase table) without
a significant loss of the ultimate translation quality as confirmed by the
automatic evaluation measure BLEU.
The differences between the baseline and \eppex{} scores were always below
0.6~points of BLEU.
In fact, \eppex{} often performed slightly better than the baseline,
although we observed that, when the phrase tables were treated with a
state-of-the-art discounting techniques, the baseline scored better.

The \emph{sigfilter} tool seemed to perform slightly better than \eppex{}
in the case of French-English translation model based on a massive training
data with only a single input factor and a language model that was simply
derived from the target side of parallel corpus.
By contrast, in the case of two-factored, Czech-English translation model with
the proper language models, a slightly better scores were obtained by \eppex{}.

Definitely the biggest advantage of \eppex{}, when compared to the baseline
system, is its runtime performance.
As \Fref{fig:conclusions-cs-en} illustrates, \eppex{} was be capable of building
the complete phrase table twice as fast as the most competitive baseline
configuration and with several times less CPU consumption.
On the other hand, \eppex{} requires a considerable amount of RAM to proceed,
while the baseline in its default configuration aims at utilizing as little RAM
as possible.
Therefore, we consider \eppex{} to be a viable alternative to the \emph{phrase-extract}
toolkit rather than a definitive replacement.

\begin{figure}[!htb]
  \centering
  \input{conclusions-cs-en}
  \caption{
    Summary of the runtime benchmarking figures for the most relevant experiments
    with the Czech-English dataset (from left to right): baseline with the default
    parameters, baseline using multiple cores and large memory buffer, \eppex{}
    with no pruning and \eppex{} with pruning configuration that achieved the best
    BLEU score from all experiments (labeled \emph{eppex defensive}).
  }
  \label{fig:conclusions-cs-en}
\end{figure}

\section{Future work}

We are aware, that in its current state, \eppex{} is a usable tool,
but a potential user of \eppex{} can be discouraged by the lack of more
finer control over its memory demands incurred during
runtime.\footnote{Without neglecting the importance of a good program design,
we cannot disregard the empirically determined principle that often the easiest
and also cheapest solution to the performance limitations imposed by hardware
is to buy a better hardware. After all, no certain amount of memory is enough
for all time.}

Addressing this possible concern, we see two possible ways of improvement:
\begin{enumerate}
  \item Provide the user with a method to safely determine the maximum amount
    of memory that \eppex{} would require to process a certain training data.
  \item Provide the user with an option to set the limit on the maximum amount
    of memory utilized by the program.
\end{enumerate}

% VM peak estimation
In \Sref{sec:eppex-memory-demands} we already sketch a possible approach of how to determine
the memory peak of epochal extraction with given pruning configuration and input data.
However, it is hard to presume whether this approach could be enhanced to offer
more precise estimation than the 20\% overestimate based on the 25\% of training
data that we achieved in our initial experiments, and how consistent this estimation
could be made.

% Iterative deepening.
The second point has two possible approaches.
The first approach is to make the program aware of its memory consumption and implement
some means of resolving the situation when all available memory is exhausted.
The second approach is to control the program memory consumption from outside:
as soon as its memory consumption exceeds predefined limit, it can be simply terminated, 
and run again with more harsh pruning (e.g. by raising the positive limit).
Obviously, the second approach does not require any changes to the existing version of
program, just a bit of scripting.

% Incremental training.
Besides the finer control over memory consumption, a yet another problem that could
be addressed in the future work on \eppex{}, is the support for \emph{incremental
training}.
With new parallel data being available every year, more and more researchers seek
the means of how to retrain their translation models without running the whole
training process from scratch.

In a way, \eppex{} already helps to work around this problem: by making the training
process much faster, the need to avoid the rerun of training process is somehow
lowered.
However, a more straight-forward solution should be also possible.
Given that phrase tables usually contain the frequency counts of phrase pairs,
these frequencies could be used to initialize the internal Lossy Counting data
memory and the epochal extraction could then proceed by processing only the new
training data.


%%% Seznam použité literatury
\bibliographystyle{csplainnat}
\bibliography{biblio}
\addcontentsline{toc}{chapter}{Bibliography}

%%% Tabulky v diplomové práci, existují-li.
%\chapwithtoc{List of tables}

%%% Použité zkratky v diplomové práci, existují-li, včetně jejich vysvětlení.
%\chapwithtoc{Seznam použitých zkratek}

%%% Přílohy k diplomové práci, existují-li (různé dodatky jako výpisy programů,
%%% diagramy apod.). Každá příloha musí být alespoň jednou odkazována z vlastního
%%% textu práce. Přílohy se číslují.

\appendix
\chapter{Installation}
\label{chap:installation}

% TODO: Pull-in eppex to the upstream Moses repository?
% \emph{Eppex} is shipped along with the Moses system and can be obtained the
% same way: by checking out Moses repository at Github.

\section*{Prerequisities}

Only Boost is required to compile and run \eppex{}.
If you had successfully compiled Moses then you probably already have all you need in place.

Just in case you did not want to install Boost as a whole, but only the necessary
parts of it, you will need to get two libraries (build them or install
them via your packaging system): \emph{Program Options} and \emph{IOStreams}.
\Eppex{} also requires working include path access to following
header-only libraries: \emph{Iterator}, \emph{Pool} and \emph{Tokenizer}.

Besides that you will probably like to employ the power of hash tables
that comes with the recent version of C++ standard.\footurl{http://en.cppreference.com/w/cpp/container/unordered_set}
Most of decent C++ compilers will have this feature available.

\section*{Download}

\emph{Eppex} is hosted on Github\footurl{http://www.github.com} in
a fork\footurl{https://github.com/chesio/mosesdecoder}
of core Moses repository\footurl{https://github.com/moses-smt/mosesdecoder}.
If you have your own local clone of Moses repo then all you need is to add
the fork as a new remote, fetch from it and checkout the remote branch
\texttt{eppex} into some new local branch.
\begin{verbatim}
 cd <your-mosesdecoder-dir>
 git remote add eppex https://github.com/chesio/mosesdecoder.git
 git fetch eppex
 git checkout -b eppex eppex:eppex
\end{verbatim}

Important note: \eppex{} branch is based on \emph{RELEASE-1.0},
not the current master. Should you like to use \eppex{} with the most
recent Moses code, you may try merging branch \texttt{master} into \texttt{eppex}:
\begin{verbatim}
 git merge master
\end{verbatim}

Be warned that you may encounter some merge conflicts in \emph{train-model.perl}
as this script gets frequently updated.
In such case feel free to contact the maintainer of \eppex{},
he will be happy to help you.

\section*{Installation}

\Eppex{} related files may be found in \texttt{contrib/eppex} directory of \eppex{} fork
and for the time being it has to be compiled separately from the Moses
build.\footnote{For the build instructions on Moses consult the file \texttt{BUILD-INSTRUCTIONS.txt}.}
Once compiled you will find \eppex{} binary (or symbolic link to it) in the
\texttt{contrib/eppex} directory. You are free to move it to whatever location
suits your working environment, just do not forget to pass the full path
to \emph{train-model.perl} script invocation.

\subsection*{Installation using Boost.Build}

This approach is recommended since \emph{Boost.Build}\footurl{http://www.boost.org/boost-build2/doc/html/index.html}
is required by Moses install process as well.
If you would like to compile \eppex{} with hash tables implementation just pass \texttt{--with-hashtables} flag
to \emph{bjam} command:
\begin{verbatim}
  cd contrib/eppex
  bjam [--with-hashtables]
\end{verbatim}

\subsection*{Installation using make}
The \texttt{contrib/eppex} directory also contains a Makefile that may be used to compile
\emph{eppex} and should work on most Linux systems that have gcc compiler installed:
\begin{verbatim}
  cd contrib/eppex
  make all
\end{verbatim}


\openright
\end{document}
