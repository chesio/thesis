% In this chapter:
% - usage
% -- counting, extracting and scoring mode
% -- missing features

\chapter{Usage}
\label{chap:usage}

Depending on your goal, you may use \eppex{} to:
\begin{itemize}
 \item collect aggregated counts of phrase pairs of same lengths
 \item generate direct and inverse extract files -- act as alternative to \emph{extract} tool
 \item construct phrase table -- act as alternative to both \emph{extract} and \emph{score} tools
\end{itemize}

For any of the three options above, \eppex{} requires three input files to proceed:
the target side of corpus, the source side of corpus and the alignment information.
Moreover, if lexical scores are to be included in the phrase translation table,
both direct and inverse lexical translation tables have to be passed.

Eppex is a drop-in replacement, therefore it expects all input files to have the same
format as expected by \emph{extract} component and produces output files in the same
format as \emph{extract} and \emph{score} components.

\section{Lossy Counting parametrization}

For each phrase pair length (or their inclusive interval) a separate Lossy Counter
instance can be created that does its "lossy counting" separately from other instances
and is fed only with phrase pairs of length following into the assigned interval.
This allows to apply different limits (or thresholds) to phrase pairs of different length.
As quality benchmarking results presented in \Cref{chap:results} suggest, it is viable
to prune short phrase pairs with milder threshold (or no threshold at all) and use harsher
thresholds only for longer phrase pairs.

A \emph{phrase pair length} is defined as the length of its longest compound:
thus, a phrase pair $(e,f)$ with $e$ being "\emph{europe}" and $f$ being
"\emph{životě v evropě}" have the length of three.
This definition is motivated by the application of maximum phrase length limit
in core \emph{extract} tool: any phrase pair having either compound of length
exceeding the limit is not extracted (ie. the phrase length maximum is decisive).

\Eppex{} provide two ways how to set \emph{support} and \emph{error} parameters
for Lossy Counting instances:
\begin{enumerate}
  \item You may specify the \emph{support} and \emph{error thresholds} directly, but doing so
    requires a good awareness of the amount of phrase pairs that is going to be fed into all
    instances of Lossy Counting algorithm, otherwise you can prune too much or do not prune at all.
  \item You may set the negative and positive limits that will directly relate to \emph{true}
    frequency counts of the phrase pairs extracted from your data. The \emph{negative limit}
    $n$ is such a value that no item with true frequency equal or less than $n$ will be output.
    The \emph{positive limit} $p$ is such a value that all items with true frequency equal
    or greater than $p$ will be output. Obviously, this forces $n < p$ (the only exception
    is the possibility to set $n = p = 0$).
\end{enumerate}

Within a single run of \eppex{} only one approach can be used.
The latter (limit-based) is much more convenient, thus recommended.
The only related draw-back is the necessity of reading the input files twice.
An additional loop over input files is done to grasp the counts of items
that will be fed to each Lossy Counter instance in order to calculate
the proper values of their \emph{support} and \emph{error} parameters,
so the properties of output will match the specified limits.
However, this counting loop runs very fast and, as a matter of fact, all the \eppex{}
experiments reported in this work have been run in such a way and all runtime
benchmarking statistics account for runs with this additional input processing loop.

\section{Command line options}

As already mentioned several times, \eppex{} is a command line tool.
The command line syntax for \emph{eppex} is following:

\begin{verbatim}
  ./eppex tgt src align [dir-lex-table [inv-lex-table]] <options>
\end{verbatim}

The 5 positional arguments are:
\begin{itemize}
 \item \verb|tgt| - path to target language corpus (required).
  May be also provided via \verb|--tgt| option.
 \item \verb|src| - path to source language corpus (required).
  May be also provided via \verb|--src| option.
 \item \verb|align| - path to alignments file (required).
  May be also provided via \verb|--align| option.
 \item \verb|dir-lex-table| - path to direct lexical table (optional).
  May be also provided via \verb|--direct-lex-table| option. 
 \item \verb|inv-lex-table| - path to inverse lexical table (optional).
  May be also provided via \verb|--inverse-lex-table| option. 
\end{itemize}

Options related to phrase pairs counting:
\begin{itemize}
 \item \verb|--counts-file <string>| - path to file with phrase pairs counts.
  Turns on counting mode, if \verb|--reuse-counts| is not specified.
 \item \verb|--reuse-counts| - use counts from the provided file instead
  of counting anew (turns off counting mode).
 \item \verb|--max-phrase-len <num>| - maximum length of extracted phrases.
  Default value is 7. If one or more phrase length values or intervals
  are set via \verb|--limits| or \verb|--thresholds|, the maximum phrase
  length is set accordingly.
\end{itemize}

Options related to phrase pairs extraction:
\begin{itemize}
 \item \verb|--extract-file <string>| - path to file for extracted phrase pairs.
  The inverse file has \texttt{.inv} extension attached to its name automatically.
  Turns on extraction mode.
 \item \verb|--thresholds <string>| - a comma separated list of \emph{error} and \emph{support}
  thresholds for Lossy Counting. For each phrase pair length or range of lengths
  a separate lossy counter can be specified, see examples below.
 \item \verb|--limits <string>| - a comma separated list of \emph{negative} and \emph{positive}
  limits for Lossy Counting. For each phrase pair length or range of lengths
  a separate lossy counter can be specified, see examples below
 \item \verb|--lambda <float>| - a value for parameter $\lambda \in (0.0, 1.0)$
  used in evaluation of the \emph{error} threshold $\epsilon$ from
  the positive limit $p$ and the negative limit $n$ as $\epsilon = (p - n - \lambda) / N$.
  Default value for $\lambda$ is 0.5.
 \item \verb|--no-final-pruning| - do not prune items with $f < (s - \epsilon)N$
  at the end of input processing, dump out all items remaining in storage.
\end{itemize}

Options related to phrase pairs scoring:
\begin{itemize}
 \item \verb|--phrase-table-file <string>| - path to phrase table file.
  Turns on scoring mode.
 \item \verb|--direct-lex-table <string>| - path to direct lexical table.
  If given, direct lexical score $lex(e|f)$ will be included in phrase table.
  May be passed also as 4-th positional parameter.
 \item \verb|--inverse-lex-table <string>| - path to inverse lexical table.
  If given, inverse lexical score $lex(f|e)$ will be included in phrase table.
  May be passed also as 5-th positional parameter.
 \item \verb|--OnlyDirect| - print only direct scores $p(e|f)$ and $lex(e|f)$
 \item \verb|--NoLex| - do not include lexical scores.
  It is sufficient to omit specification of lexical table files to not
  include lexical scores, but this switch can be used to override their presence.
 \item \verb|--NoPhraseCount| - do not include phrase counts.  
 \item \verb|--NoWordAlignment| - do not print word alignment.
 \item \verb|--GoodTuring| - adjust phrase translation probabilities with
  Good-Turing discounting
 \item \verb|--KneserNey| - adjust phrase translation probabilities with
  Kneser-Ney discounting
 \item \verb|--LogProb| - print logarithm of probabilities
 \item \verb|--NegLogProb| - print negative logarithm of probabilities
\end{itemize}

Options related to both phrase pairs extraction and scoring:
\begin{itemize}
 \item \verb|--GZOutput| - gzip the output (automatically
  adds \texttt{.gz} extension to the filenames provided).
\end{itemize}

It is important to mention that at least one of the options providing an output
file has to be set (\verb|--counts-file|, \verb|--extract-file| or \verb|--phrase-table-file|),
otherwise \eppex{} will be unable to decide on how to proceed.

Finally, three program options exist that provides some information about the program:
\begin{itemize}
 \item \verb|--help| - prints the program help.
 \item \verb|--info| - prints basic information about the program.
 \item \verb|--version| - prints program version.
\end{itemize}

\section{Missing features}

In the moment eppex cannot extract orientation info data required for training
of reordering models.

Also some of the features of the core \emph{phrase-extract} suite from Moses
release 1.0 are yet missing in \eppex{}, namely:
\begin{itemize}
  \item inclusion of sentence ID in extract files: see \verb|--IncludeSentenceId|
    option of \emph{extract} tool
  \item adding sentence weights to extracted phrases: see \verb|--InstanceWeights|
    option of \emph{extract} tool
  \item dumping singleton feature: see \verb|--Singleton| option of \emph{scorer} tool
  \item adding penalty to unaligned (function) words: see \verb|--UnalignedPenalty|
    and \verb|--UnalignedFunctionWordPenalty| options of \emph{scorer} tool
\end{itemize}

\section{Examples}

Let us finish this chapter with some examples that demonstrate some typical
patterns of \eppex{} usage. Because input files specification is always required,
we will only mark it by \verb|<input>| in command line snippets below.

Grab counts of phrase pairs of lengths up to 8 that will be extracted from
corpus and store it in file \texttt{counts.txt}:
\begin{verbatim}
  eppex <input> --max-phrase-len 8 --counts-file counts.txt
\end{verbatim}

Construct phrase table reusing the counts from file \texttt{counts.txt} and using
two Lossy Counting instances: one to keep all phrase pairs with length from
1 to 3 and another to remove all phrase pairs with length from 4 to 6 with true
frequency equal to 1 and keep all with true frequency at least 3. Despite the
\texttt{counts.txt} file contains counts for phrase pairs lengths up to 8,
the specification of \verb|--limits| determines that phrase table will contain
only phrase pairs of length up to 6:
\begin{verbatim}
  eppex <input> --counts-file counts.txt --reuse-counts \\
    --limits 1-3:0:1,4-6:1:3 --phrase-table-file phrase-table
\end{verbatim}

Construct phrase table with phrase pairs pruned according to the following rule:
for each phrase pair length $L$ only phrase pairs with true frequency at least
$L$ will be retained, the rest might or might not be pruned depending on the
distribution of their occurrences in parallel corpus:
\begin{verbatim}
  eppex <input> --phrase-table-file phrase-table \\
    --limits 1:0:1,2:0:2,3:0:3,4:0:4,5:0:5,6:0:6,7:0:7
\end{verbatim}

Construct complete phrase table not removing a single phrase pair.
The phrase table will be stored gzipped in file \texttt{phrase-table.gz}
and will contain phrase pairs of length up to (default maximum) 7.
\begin{verbatim}
  eppex <input> --phrase-table-file phrase-table --GZOutput
\end{verbatim}
