% In this chapter:
% - usage
% -- counting, extracting and scoring mode
% -- missing features

\chapter{Usage}
\label{chap:usage}

As mentioned above, \eppex{} is a command line tool.

\Eppex{} requires three input files to proceed:
the target side of corpus, the source side of corpus and the alignment information.
Depending on your goal, you may use \eppex{} to:
\begin{itemize}
 \item collect aggregated counts of phrase pairs of same lengths
 \item generate direct and inverse extract files (as alternative to \emph{extract} tool)
 \item construct phrase table (as alternative to both \emph{extract} and \emph{score} tools)
\end{itemize}

Eppex is a drop-in replacement, therefore it must be passed input files
in the same format as expected by \emph{extract} component and
the output file it produces conforms to the format of \emph{score} component.

\section{Terminology}

% Introduce the definitions of:
% - phrase pair length
% - positive and negative lossy counting limits
% - lambda parameter

A phrase pair length is defined as the length of its longest compound.

\section{Command line options}

It can be used to produce the following output:
\begin{itemize}
 \item \emph{counts file} - file with frequency counts for each phrase pair
  length up to given limit (set via \verb|--max-phrase-len| option)
 \item \emph{extract files} - direct and inverse extract files. These files are the
  same as the output files produced in step 5 of \emph{train-model.perl} pipeline,
  except that the frequency count of each phrase pairs is not determined by
  the number of times it occurs in the files but for space efficiency it is
  dumped as an additional, 4-th field of each phrase pair record (this format
  is properly recognized by \emph{phrase-extract/scorer}).
 \item \emph{phrase table file} - the ultimate phrase table.
\end{itemize}

The command line syntax for \emph{eppex} is following:

\begin{verbatim}
  ./eppex tgt src align [dir-lex-table [inv-lex-table]] <options>
\end{verbatim}

The positional arguments are:
\begin{itemize}
 \item \verb|tgt| - path to target language corpus (required).
  May be also provided via \verb|--tgt| option.
 \item \verb|src| - path to source language corpus (required).
  May be also provided via \verb|--src| option.
 \item \verb|align| - path to alignments file (required).
  May be also provided via \verb|--align| option.
 \item \verb|dir-lex-table| - path to direct lexical table (optional).
  May be also provided via \verb|--dir-lex-table| option. 
 \item \verb|inv-lex-table| - path to inverse lexical table (optional).
  May be also provided via \verb|--inv-lex-table| option. 
\end{itemize}

Options related to phrase pairs counting:
\begin{itemize}
 \item \verb|--counts-file| - path to file with phrase pairs counts.
 \item \verb|--reuse-counts| - use counts from the provided file instead
  of counting anew (skips counting mode).
 \item \verb|--max-phrase-len| - maximum length of extracted phrases.
\end{itemize}

Options related to phrase pairs extraction:
\begin{itemize}
 \item \verb|--extract-file| - path to file for extracted phrase pairs.
  The inverse file has $.inv$ extension attached to its name automatically.
 \item \verb|--thresholds| - a space separated list of \emph{error} and \emph{support}
  thresholds for lossy counting. For each phrase pair length or range of lengths a separate lossy counter
  can be specified, for example: 1:0.5:1.5 2-3:1.0:2.0 4-7:3.0:5.0
 \item \verb|--limits| - a space separated list of \emph{negative} and \emph{positive}
  limits for lossy counting. For each phrase pair length or range of lengths a separate lossy counter
  can be specified, for example: 1:0.5:1.5 2-3:1.0:2.0 4-7:3.0:5.0
 \item \verb|--lambda| - the value for lambda parameter, real number from
  (0.0, 1.0) interval. Default value is 0.5.
\end{itemize}

Options related to phrase pairs scoring:
\begin{itemize}
 \item \verb|--phrase-table-file| - path to phrase table file.
  The inverse file has $.inv$ extension attached to its name automatically.
 \item \verb|--dir-lex-table| - path to direct lexical table.
  If given, direct lexical score $lex(e|f)$ will be included in phrase table.
  May be passed also as 4-th positional parameter.
 \item \verb|--inv-lex-table| - path to inverse lexical table.  
  If given, inverse lexical score $lex(f|e)$ will be included in phrase table.
  May be passed also as 5-th positional parameter.
 \item \verb|--OnlyDirect| - print only direct scores $p(e|f)$ and $lex(e|f)$
 \item \verb|--NoLex| - do not include lexical scores.
  It is sufficient to omit \verb|--direct-lex-table| and \verb|--inverse-lex-table|
  options to not include lexical scores, but this switch can be used to
  override their presence.
 \item \verb|--NoPhraseCount| - do not include phrase counts.  
 \item \verb|--NoWordAlignment| - do not print word alignment.
 % TODO: Finish.
\end{itemize}

Options related to both phrase pairs extraction and scoring:
\begin{itemize}
 \item \verb|--GZOutput| - gzip the output (automatically
  adds $.gz$ extension to the filenames provided).
\end{itemize}

Finally, three program options that provides some information about the program:
\begin{itemize}
 \item \verb|--help| - prints the program help.
 \item \verb|--info| - prints basic information about the program.
 \item \verb|--version| - prints program version.
\end{itemize}

\section{Missing features}

In the moment eppex cannot extract orientation info data required for reordering models training.

Also some of the recently added features to the legacy scoring tool are yet missing in \eppex{},
namely:
\begin{itemize}
  \item smoothing options available in legacy scorer: \verb|--GoodTuring| and \verb|--KneserNey|
  \item dumping singleton feature available in legacy scorer: \verb|--Singleton|
  % TODO: And several others - some of them are only viable for hierarchical models.
\end{itemize}
